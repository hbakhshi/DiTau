\section{Review of the experimental analysis}
This study reinterprets the results of the ``Search for electroweak production of charginos in final states with two tau leptons in pp collisions at sqrt(s) = 8 TeV''\cite{Khachatryan:2016trj} for the production of two \wprime bosons decaying into two $\tau$ leptons. In this section the experimental analysis is reviewed.

The $\tau$ lepton decays  to a muon or an electron in $\sim 35\%$ of the cases and to hadrons (\Tau) in the rest of cases. So about 90\% of events with two $\tau$ leptons decay to \lepTau ~or \tauTau. In this analysis, only these two decay channels are considered and the fully leptonic decay modes are ignored. 

The missing transverse momentum (\MET) which is  defined as the vectorial sum of the transverse momenta (\pt) of neutrinos in the event is one of the event properties used in this analysis. In addition to the neutrinos  produced from the direct decay of \wprime boson, those produced in the $\tau$ lepton decay are also considered.

Having the momentum of the decay products of $\tau$ leptons and \MET, we can calculate all the needed variables used in the reference analysis. Stransverse mass ($M_{T2}$)~\cite{Lester:1999tx,Barr:2003rg}  is the main variable that is used to categorize the events. It is a function of momentum of two visible particles and \MET ~in the event. It is defined as:
\begin{equation}
M^2_{T2}(m_{N},\alpha,\beta,\MET) = \displaystyle\min_{p_T+q_T=\MET} [\max[M^2_{T}(\alpha,p),M^2_{T}(\beta,q)]]
\end{equation}
Where $\alpha$ and $\beta$ ($p$ and $q$) are the four momenta of the visible (invisible) decay products in two different legs and  $m_N$ is the mass of the invisible particle which is set to zero for this study. The transverse mass ($M_{T}$) is defined as :
\begin{equation}
M^2_{T}(\alpha,p) =  m^2_{\alpha}+m^2_N+2(E_T(p)E_T(\alpha)-\vec{p}_T.\vec{\alpha}_T)
\end{equation}
and transverse energy is given by; 
\begin{equation}
E_T(p)=\sqrt{p^2_T+m^2_N}
\end{equation}

In the reference analysis, the events are categorized in 4 signal regions (SR). For the e\Tau ~and $\mu\Tau$, it has been found that cutting the events with $\mttwo <$ 90 GeV is useful to discard the SM background events. But for \tauTau ~events, two separate SR's are defined as events with $\mttwo>$ 90 GeV (SR1) and events with 40 $<\mttwo<$ 90 GeV  and \SumMT $>$ 250 GeV (SR2), where \SumMT is the sum of the transverse mass of two \Tau ~objects.

In addition to the selection criteria that are applied for the lepton selection and event categorization, some other requirements to veto any extra lepton or b-tagged jets  are also applied. 

