\section{Review of the experimental analysis}
% {\bf The text should be updated to address the following items discussed in the meeting.}
% 3 [Bakhshians] Review of experimental results
% [X] 3.1 4channels
% [X] 3.1.1 MT2 is used, then explain about MT2
% [ ] 3.2 Efficiencies are reported
% [ ] 3.2.1 how they can be used


This study reinterprets the results of the results of the ``Search for electroweak production of charginos in final states with two tau leptons in pp collisions at sqrt(s) = 8 TeV''\ref{Khachatryan:2016trj} for the production of two \wprime s decaying into two $\tau$ leptons. In this section the experimental analysis is reviewed.

The $\tau$ lepton decays to leptons (either to a $\mu$ or an electron) in $\approx 35\%$ of the cases and to hadrons in the rest of cases. So $\approx 90\%$ of events with two $\tau$ leptons decay to $\ell\tau_h$ or $(\tau_h \tau_h)$. In this analysis, only these two decay channels are considered and the fully leptonic decay modes are ignored. For the decay of the $\tau$, the TAUOLA(REF) package has been used. More details are described in section 3(REF).

Missing transverse energy (MET) is also used for this study. This event propery has been measured as the vectorial sum of the transverse momentums ($p_{T}$) of neutrinos in the event. In addition to the neutrinos that are produced from the direct decay of \wprime, the decay products of the $\tau$ lepton are also considered.

Having the momentum of the decay products of $\tau$ leptons and the MET, we can calculate all the needed varaiables used in the reference analysis. Stransverse mass ($M_{T2}$)(REF) is the main variable that is used to categorize events. It is a function of momenta of two visible particles and the missing transverse momentum in the event. It is defined as [ref];
\begin{equation}
m^2_{T2}(m_{N},\alpha,\beta,\not\!P_{T}) = \displaystyle\min_{p_T+q_T=\not\!P_T} [max[m^2_{T}(\alpha,p),m^2_{T}(\beta,q)]]
\end{equation}
Where $\alpha$ and $\beta$ are the four momenta of the visible decay products of the $\tau$ lepton decay and  $m_N$ is the mass of the invisible particle (neutrino) which is set to zero for this study. $\not\!P_T$ is missing transverse momenta. The transverse mass ($m_{T}$) is defined as :
\begin{equation}
m^2_{T}(\alpha,p) =  m^2_{\alpha}+m^2_N+2(E_T(p)E_T(\alpha)-p_T.\alpha_T)
\end{equation}
and transverse energy is given by; 
\begin{equation}
E_T(p)=\sqrt{p^2_T+m^2_N}
\end{equation}

In the reference analysis, the events are categorized in 4 signal regions (SR). For the e$\tau_h$ and $\mu\tau_h$, it has been found that cutting the events with $M_{T2}<90 GeV$ is useful to discard the standard model background events. But for $(\tau_h \tau_h)$ events, two separate SRs are defined : events with $M_{T2}>90 GeV$ or events with $M_{T2}<90 GeV$ and $\sum m_{T} > 200 GeV$ where $\sum m_{T}$ is the sum of the transverse mass of two $\tau$ leptons.

In addition to the selection cuts that are applied for the lepton selection and event categorization, some other cuts to veto any extra lepton or b-tagged jet  are also applied. 



 % We utilized ExRootAnalysis [ref] for making root files after generation. For calculating efficiency we modelled \cite{CMS:2016saj} for using result of applied cuts and seprating signal region to four sections: SR1 and SR2 which are relate to fully hadronic final state and also e$\tau$ and $\mu\tau$ which are relate to single lepton final state. As much as SM background events that mainly produced by W+jets, intended to have $M_{T2}<90GeV$, the distribution of $M_{T2}$ in first signal region SR1 demands $M_{T2}>90GeV$. however, the second signal region SR2 requires sum of two transverse mass of visible particles to take $\sum{m^{\tau(i)}_T}=(m^{\tau_{1h}}_T + m^{\tau_{2h}}_T>250GeV)$ in addition of lower $M_{T2}$ cut $M_{T2}<90GeV$ to distinguish signal and background process in the case that there is a small mass difference ($\Delta m$) between W' and neutrino.The former cut, moreover, can veto b-tagged jets produced in $t{\bar{t}}$ events in law $M_{T2}$. 
 % for rejecting W+jets and b-tagged backgrounds in $\ell\tau$($\mu\tau  or  e\tau$) channels, $M_{T2}>90GeV$ and $m^{\tau_{h}}_T>200 GeV$ have been applied. Despite fully hadronic channel SR2, the cut $M_{T2}<90GeV$ has not been used, because of higher level of background. The other event selections set to suppress backgrounds such as $t{\bar{t}}$+jets,Drell-yan and QCD mutijets and Z+jets and etc which based on data in[ref] list below. we abbrevited selection requirement in different channels. Items in $\ell\tau$ channels:
 
 % \begin{itemize}
 % 	\item[-] $P_T>20GeV , |\eta|<2.3$ for isolated $\tau_{h}$ as well as $|\eta|<2.1$ for isolated $e or \mu$
 % 	\item[-] $P_T>10GeV$ to supress events with additional lepton in Zboson decays
 % 	\item[-] $15<{MT_{\ell{\tau_h}}}<45GeV$ to reject b-tagged jets 
 % \end{itemize}
 
 % and items in $\tau_{h}  \tau_{h}$ channels:
 
 % \begin{itemize}
 % 	\item[-] $P_T>45GeV , |\eta|<2.1$ and have oposite charge for $\tau$ isolation discrimination
 % 	\item[-] $P_T>10GeV , |\eta|<2.4$ to suppress events with extra $e or \mu$ in diboson decays
 % 	\item[-] $15<{MT_{di-{\tau_h}}}<45GeV$ for Z veto
 % 	\item[-] $P^{miss}_T>30GeV, M_{T2}>40GeV$ for Zveto and rejecting QCD multijets 
 % 	\item[-] $MT_{\tau_h \tau_h}>15GeV$ for discarding Drell-yan productions and QCD mutijets
 % 	\item[-] $\Delta\phi>1$ for excluding QCD mutijets and W+jets
 % \end{itemize}
 
 % The calculation of total efficiencies in different signal regions that introduced above has been performed by using effecienies for each variable which measured due to the requirement cuts in various channels from [ref].The results have been displayed in table II.
 
 % \begin{table}[htb]
 % 	\centering
 % 	\begin{tabular}{|ccccc|}
 % 		\hline 
 % 		Mass Wprime  & EfficiencySR1  & EfficiencySR2 & Efficiency $\mu$$\tau$ & Efficiency e $\tau$ \\\hline 
 % 		100 & 0.0458899 \% & 0.614535 \% & 0.011183 \% & 0.00395297 \% \\
 % 		130& 0.175685 \%& 0.760096 \%& 0.0442253 \%& 0.0271803 \% \\
 % 		160& 0.410097 \%& 0.846774 \%& 0.135277 \%& 0.0756016 \%\\
 % 		190& 0.715545 \%& 0.909511 \%& 0.228263 \%& 0.189238 \% \\
 % 		220& 1.06748 \%& 0.968385 \%& 0.43919 \%& 0.331424 \% \\
 % 		250& 1.47652 \%& 0.983356 \%& 0.632992 \%& 0.516825 \%\\ 
 % 		280& 1.80193 \%& 1.02289 \%& 0.861038 \%& 0.703542 \% \\
 % 		310& 2.14124 \%& 1.05743 \%& 1.12391 \%& 0.865479 \% \\
 % 		340& 2.45444 \%& 1.03443 \%& 1.46279 \%& 1.18334 \%\\ 
 % 		370& 2.76015 \%& 1.00684 \%& 1.6293 \%& 1.38178 \% \\
 % 		400 &2.99298 \%& 0.97716 \% &1.87191 \%& 1.62458 \% \\
 % 		\hline
 % 	\end{tabular}
 % 	\caption{Cross sections and decay widths corresponding to various W' masses. \label{tab1} }
 % \end{table}
 