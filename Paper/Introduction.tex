%\section{THEORICAL EXPLANATION}\label{sec:cone}
\section{Introduction}\label{sec:int} 


{\bf The text should be updated to address the following items discussed in the meeting.}

2 [Dr. Paktinat] Introduction

2.1 Theorical explanation

2.1.1 Wprime introduction

right-left handed

2.1.2 experimental results

2.1.3 motivate DiWPrime production

2.1.4 Motivate to tau decay

2.2 explain this study

2.2.1 using the cms ditau results and interpret it




New heavy gauge bosons, called wprime, is predicted by numerous different extensions of standard model. Surveys of wprime have been done at Tevatron [ref]several times and it started at large hadron collider (LHC) in 2010 [ref] in various leptonic and hadronic channels. W boson couples to only left-handed fermions,whereas, the coupling of wprime boson could be compeletly left-handed or compeletly right-handed or mix of both. It is predictable that wprime boson couple strongly with third generation of quarks. In the other hand, suppressing multijet backgrands for third generation is easier than other generation, so, these reasons made this channel more importance for studying.
one of the extension of standdard model for producing interactions of fermions to a w boson is wprime effective model [ref]. The lagrangian term which describes coupling of wprime boson given by[ref];
\begin{eqnarray}
{\cal L}& =& \frac{V_{ij}}{2\sqrt{2}}\bar{f_i} \gamma_{\mu}(g^\prime_{R_{i,j}} (1+{\gamma}^5)+
g^\prime_{L_{i,j}}
(1-{\gamma}^5)) W^{\prime \mu} f_j  \nonumber\\
&+& \mathrm{h.c.},
\end{eqnarray}
where $g'_R(L)_{i,j}$ are right handed (left-handed) coupling constants. $V'_{i,j}$ reffers to CKM matrix and $(1\pm{\gamma^5})$ represents left-handed (right-handed) chiral projection operators. In the case: $g'_R = 0$ and $g'_L = g_{(SM)}$ (right-handed decays) both leptons and hadrons and where $g'_R = g_{(SM)}$ and $g'_L = 0$ (left-handed decays) only leptons are produced.If the mass of W' were lighter than right-handed neutrino, only hadronic decay of W' is allowed.
In this paper, two W' bosons decay to a tau and its neutrino ($\tau$ and $\nu_{\tau}$). tau maybe decays hadronic $(\tau_h  \tau_h)$ or leptonic $(\tau_h \ell)$. This analysis' searches for W' boson is performed at LHC in proton-proton collisions at a center-of-mass energy of $\sqrt{8}TeV$, corresponding to an integrated luminosity of 18.1 and 19.6 $fb^{-1}$ in different channels. Final state considered once fully hadronic $(\tau_h \tau_h)$ and again where one tau decays hadronic and other tau decays leptonic.The schematic diagram of decay is shown in figure 1.