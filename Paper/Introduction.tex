\section{Introduction}\label{sec:int} 
New heavy charged gauge bosons, called \wprime bosons, are predicted by numerous different extensions of the standard model of the elementary particles (SM). 
The SM $W$ boson couples only to left-handed fermions, whereas the coupling of \wprime boson can be completely left-handed, right-handed or a mixture of both. 

The general form of the lagrangian describing the fermionic interactions of \wprime boson is given in  Ref.~\cite{Sullivan:2002jt}
\begin{eqnarray}
{\cal L}& =& \frac{V_{ij}}{2\sqrt{2}}\bar{f_i} \gamma_{\mu}(g^\prime_{R} (1+{\gamma}^5)+
g^\prime_{L}
(1-{\gamma}^5)) W^{\prime \mu} f_j  \nonumber\\
&+& \mathrm{h.c.},
\label{eq:lagrangian}
\end{eqnarray}
where $g'_{R(L)}$ are the right handed (left-handed) coupling constants. The $V_{ij}$ matrix refers to a $3\times3$ identity matrix for leptons and the CKM matrix for quarks. The $(1\pm{\gamma^5})$ operators represent left and right-handed chiral projection operators. In the case, \gR = 0 and \gL $\neq$ 0 (pure left-handed), both leptons and quarks can couple to \wprime boson, but where \gR $\neq$ 0 and \gL = 0 (pure right-handed) only quarks can couple to \wprime boson, because we either do not introduce right-handed neutrinos or they are assumed to be much heavier than \wprime boson. 

In this paper, for the first time, we consider a situation where two opposite-sign \wprime bosons are produced. Since nowadays the colliders center of mass energy is sufficiently high, such processes can be accessible. Due to the important role of the third generation fermions in many new physics scenarios, each \wprime boson is decayed to a $\tau$ lepton and its neutrino ($\nu_{\tau}$). Since we ask for two $\tau$ leptons in the final state, $g'_L$ can not be zero. 

In this analysis, the efficiencies provided by the CMS experiment \cite{Khachatryan:2016trj} are used to find the yields of the favorite signal and compare it with the reported SM backgrounds to set a lower limit on the mass of \wprime boson. 
The CMS analysis uses LHC data from proton-proton (pp) collisions at a center-of-mass energy ($\sqrt{s}$) of 8 TeV to search for new physics in di-tau final states.  The data corresponds to an integrated luminosity of 18.1 and 19.6 $fb^{-1}$ in different channels. Three different final states are considered depending on the decay of two $\tau$ leptons, fully hadronic (\tauTau), where both $\tau$ leptons decay hadronically and \lepTau  ~(e\Tau ~or $\mu\Tau$), where one $\tau$ lepton decays hadronically and the other decays leptonically. The schematic diagram of decay is shown in figure \ref{fig:wprimefeyndiagram}.
\begin{figure}[!htb]
  \includegraphics*[width=.45\textwidth]{figs/WpWpTauTau.pdf}
  \caption{The diagram showing the production process of two \wprime bosons in collision of two protons and their decay into $\tau$ and $\nu_{\tau}$. The $\tau$ lepton decays either to lighter leptons, i.e. electron and muon, or to hadrons. For the analysis presented in this paper, at least one of the $\tau$ leptons decay hadronically.}
  \label{fig:wprimefeyndiagram}
\end{figure}

In different experiments, many searches are done to see the signatures of \wprime boson, but up to now all of them have failed. The most stringent limit is set by the ATLAS experiment in an analysis which looks at the tail of the lepton transverse mass distribution \cite{Aaboud:2017efa}. The lepton is assumed to be produced in the decay of a \wprime boson associated with missing transverse energy coming from a neutrino. It has excluded the \wprime boson with masses smaller than 5.1 TeV at a 95\% confidence level (CL), assuming a pure left-handed \wprime boson with \gL ~equal to the coupling of the SM $W$ boson (\gSM = 0.64). The results of different direct searches from the colliders are also used to constrain the \wprime boson. For example, the results of the search for single top quark production are used to constrain the \wprime boson in Ref. \cite{YaserAyazi:2017xyj}.


In next section, the reference experimental analysis is reviewed and the used variables are defined. In Section \ref{sec:simulation}, the framework of our analysis is described. The results of the analysis are reported in Section \ref{sec:results}. Section \ref{sec:conclusion} concludes the paper.