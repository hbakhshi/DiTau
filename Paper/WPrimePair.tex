
%\documentclass[twocolumn,showpacs,preprintnumbers,amsmath,amssymb]{revtex4}
%
%\documentclass[twocolumn,english,showpacs,preprintnumbers,amsmath,amssymb,floatfix]{revtex4}
%\documentclass[preprint,showpacs,preprintnumbers]{revtex4}
\RequirePackage{lineno}
%\usepackage{lineno}
\documentclass[preprint,showpacs,preprintnumbers]{revtex4}

% Some other (several out of many) possibilities
%\documentclass[preprint,aps]{revtex4}
%\documentclass[preprint,aps,draft]{revtex4}
%\documentclass[prb]{revtex4}% Physical Review B
\newcommand{\wprime}{\ensuremath{W^\prime}~}
\newcommand{\wprimep}{\ensuremath{W^{\prime+}}}
\newcommand{\wprimem}{\ensuremath{W^{\prime-}}}
\newcommand{\Tau}{\ensuremath{\tau_h}}
\newcommand{\tauTau}{\ensuremath{\tau_h\tau_h}}
\newcommand{\lepTau}{\ensuremath{\ell\tau_h}}
\newcommand{\pt}{\ensuremath{p_T}}
\newcommand{\mttwo}{\ensuremath{M_{T2}}}
\newcommand{\MET}{\ensuremath{p_T^{miss}}}
\newcommand{\SumMT}{ \ensuremath{\Sigma M_\mathrm{T}^{\tau_i}}~}

%\newcommand{\MPT}{\ensuremath{P_T\hspace{-1.1em}/\kern 1.1em}\xspace}
%------------------------------------------------------
\usepackage{graphicx}% Include figure files
%\usepackage{subfig}
\usepackage{dcolumn}% Align table columns on decimal point
\usepackage{bm}% bold math
\usepackage{epsfig}
\usepackage{epstopdf}
\usepackage{grffile}
\usepackage{color}
\usepackage{colordvi}
\usepackage{amsmath,amssymb}
\usepackage{rotating}
\usepackage{lscape}
\usepackage{float}


\usepackage{footnote}
 \makesavenoteenv{environmentname}


\usepackage{hyperref}

\usepackage[T1]{fontenc} % if needed



\begin{document}


\linenumbers

\title{$W'$ Pair Production in the Light of CMS Searches}



\author{Hamed Bakhshian}
\email{hamed.bakhshian@cern.ch}
\affiliation{UCL}

\author{Saeid Paktinat Mehdiabadi}
\email{paktinat@ipm.ir}
\affiliation{School of Particles and Accelerators, Institute for Research in Fundamental Sciences (IPM), P.O.Box 19395-5531, Tehran, Iran\\
Faculty of Physics, Yazd University, P.O. Box 89195-741, Yazd, Iran}

\author{Leila Zamiri}
\email{leila.zamiri@gmail.com}
\affiliation{Independent Researcher???}





\date{\today}

\begin{abstract}
For the first time, the pair production of the heavy charged gauge bosons, known as $W'$ boson is considered, when both decay to $\tau$ leptons. The reported detailed efficiency of object/event selection in the CMS experiment is used to find the lower limit on the mass of $W'$ boson. Various assumptions for the coupling of the new gauge boson are examined and the results are reported. In the case of a SM-like $W'$ boson, masses below ... GeV are excluded at 95\% confidence level. The method can be used to constrain the new models with the similar final state.
 \keywords{heavy gauge boson, collider phenomenology, exclusion limits}
\end{abstract}


\pacs{}
\maketitle


%\section{THEORICAL EXPLANATION}\label{sec:cone}
\section{Introduction}\label{sec:int} 


{\bf The text should be updated to address the following items discussed in the meeting.}

2 [Dr. Paktinat] Introduction

2.1 Theorical explanation

2.1.1 Wprime introduction

right-left handed

2.1.2 experimental results

2.1.3 motivate DiWPrime production

2.1.4 Motivate to tau decay

2.2 explain this study

2.2.1 using the cms ditau results and interpret it




New heavy gauge bosons, called wprime, is predicted by numerous different extensions of standard model. Surveys of wprime have been done at Tevatron [ref]several times and it started at large hadron collider (LHC) in 2010 [ref] in various leptonic and hadronic channels. W boson couples to only left-handed fermions,whereas, the coupling of wprime boson could be compeletly left-handed or compeletly right-handed or mix of both. It is predictable that wprime boson couple strongly with third generation of quarks. In the other hand, suppressing multijet backgrands for third generation is easier than other generation, so, these reasons made this channel more importance for studying.
one of the extension of standdard model for producing interactions of fermions to a w boson is wprime effective model [ref]. The lagrangian term which describes coupling of wprime boson given by[ref];
\begin{eqnarray}
{\cal L}& =& \frac{V_{ij}}{2\sqrt{2}}\bar{f_i} \gamma_{\mu}(g^\prime_{R_{i,j}} (1+{\gamma}^5)+
g^\prime_{L_{i,j}}
(1-{\gamma}^5)) W^{\prime \mu} f_j  \nonumber\\
&+& \mathrm{h.c.},
\end{eqnarray}
where $g'_R(L)_{i,j}$ are right handed (left-handed) coupling constants. $V'_{i,j}$ reffers to CKM matrix and $(1\pm{\gamma^5})$ represents left-handed (right-handed) chiral projection operators. In the case: $g'_R = 0$ and $g'_L = g_{(SM)}$ (right-handed decays) both leptons and hadrons and where $g'_R = g_{(SM)}$ and $g'_L = 0$ (left-handed decays) only leptons are produced.If the mass of W' were lighter than right-handed neutrino, only hadronic decay of W' is allowed.
In this paper, two W' bosons decay to a tau and its neutrino ($\tau$ and $\nu_{\tau}$). tau maybe decays hadronic $(\tau_h  \tau_h)$ or leptonic $(\tau_h \ell)$. This analysis' searches for W' boson is performed at LHC in proton-proton collisions at a center-of-mass energy of $\sqrt{8}TeV$, corresponding to an integrated luminosity of 18.1 and 19.6 $fb^{-1}$ in different channels. Final state considered once fully hadronic $(\tau_h \tau_h)$ and again where one tau decays hadronic and other tau decays leptonic.The schematic diagram of decay is shown in figure 1.

\section{Review of the experimental analysis}
% {\bf The text should be updated to address the following items discussed in the meeting.}
% 3 [Bakhshians] Review of experimental results 
% [X] 3.1 4channels {\bf Done}
% [X] 3.1.1 MT2 is used, then explain about MT2 {\bf Done}
% [ ] 3.2 Efficiencies are reported 
% [ ] 3.2.1 how they can be used


This study reinterprets the results of the ``Search for electroweak production of charginos in final states with two tau leptons in pp collisions at sqrt(s) = 8 TeV''\cite{Khachatryan:2016trj} for the production of two \wprime bosons decaying into two $\tau$ leptons. In this section the experimental analysis is reviewed.

The $\tau$ lepton decays to leptons (either to a $\mu$ or an electron) in $\sim 35\%$ of the cases and to hadrons (\Tau) in the rest of cases. So about 90\% of events with two $\tau$ leptons decay to \lepTau or \tauTau. In this analysis, only these two decay channels are considered and the fully leptonic decay modes are ignored. 
%For the decay of the $\tau$ lepton, the TAUOLA~\cite{Jadach:1993hs} package has been used. More details are described in section 3(REF).

The missing transverse momentum (\MET) which is  defined as the vectorial sum of the transverse momenta (\pt) of neutrinos in the event is one of the event properties used in this analysis. In addition to the neutrinos  produced from the direct decay of \wprime, those produced in the $\tau$ lepton decay are also considered.

Having the momentum of the decay products of $\tau$ leptons and \MET, we can calculate all the needed varaiables used in the reference analysis. Stransverse mass ($M_{T2}$)~\cite{Lester:1999tx,Barr:2003rg}  is the main variable that is used to categorize the events. It is a function of momentum of two visible particles and \MET ~in the event. It is defined as:
\begin{equation}
M^2_{T2}(m_{N},\alpha,\beta,\MET) = \displaystyle\min_{p_T+q_T=\MET} [\max[M^2_{T}(\alpha,p),M^2_{T}(\beta,q)]]
\end{equation}
Where $\alpha$ and $\beta$ ($p$ and $q$) are the four momenta of the visible (invisible) decay products in two different legs and  $m_N$ is the mass of the invisible particle which is set to zero for this study. The transverse mass ($M_{T}$) is defined as :
\begin{equation}
M^2_{T}(\alpha,p) =  m^2_{\alpha}+m^2_N+2(E_T(p)E_T(\alpha)-\vec{\pt}.\vec{\alpha_T})
\end{equation}
and transverse energy is given by; 
\begin{equation}
E_T(p)=\sqrt{p^2_T+m^2_N}
\end{equation}

In the reference analysis, the events are categorized in 4 signal regions (SR). For the e\Tau ~and $\mu\Tau$, it has been found that cutting the events with $\mttwo <$ 90 GeV is useful to discard the SM background events. But for \tauTau ~events, two separate SR's are defined as events with \mttwo $>$ 90 GeV and events with 40 $<\mttwo<$ 90 GeV and \SumMT $>$ 250 GeV, where \SumMT is the sum of the transverse mass of two \Tau ~objects.

In addition to the selection criteria that are applied for the lepton selection and event categorization, some other requirements to veto any extra lepton or b-tagged jets  are also applied. 



\section{SIMULATION}\label{sec:evo}

{\bf The text should be updated to address the following items discussed in the meeting.}

4 [Zamiri] Simulation of signal events

4.1 madgraph

4.2 reference to the wprime package that used

4.3 quote coupling value

4.3.1 doubled and halfed

4.4 explain about the mixture of lef-and right handed

4.5 decay width and validation (ref to the W->tb)

4.6 cross sections (mass, mixture angle)

4.7 discussion about the branching fractions

4.7.1 assume SM values

4.7.2 assume that it prefers tau between leptons

4.8 validation plots ??? / kinematic distributions

4.8.1 gen mt2 for different masses

4.8.2 gen tau-pt for different masses

4.9 yields and final efficiencies

4.9.1 2 tables for SM coupling and double coupling (for different masses)

4.9.2 1 table for SM coupling and different mixing angles = (0,30,45,60) (for 
mass=220GeV)



{\small MADGRAPH 5} is the new version of {\small MADGRAPH} matrix generator [ref]. We used MadGraph [ref] for simulating signal and backgrands events beacause implementation of accurate simulation for both signals and backgrounds is based on the ability of {\small MADGRAPH}. It is the tool with stronge power of extracting physics from the data and making precise prediction for physics Beyond the Standard Model {\small (BSM)}. Generating tree-level matrix elements and measurement of cross sections for classes of renormalizable models and establishment of {\small FEYNRULES} [ref] to perform new physics models and finaly making links between {\small FEYNRULES} and matrix elements and standardizing them by Universal F{\small EYNRULES} Output format, the {\small (UFO)} [ref], are some of the significant benefits of {\small MADGRAPH 5}.

The model which performed for signal generation was $W'$ Effective model {\small (WEff-UFO)}, that is an extension of Standard Model {\small (SM)}. {\small UFO} is the flexible format of models goes beyond existing formats in order to provide the abstract information about the particles, parameters and vertices of the model that can be translate into {\small PYTHON} [ref] and achieve by matrix-element generators. 

As it is mentioned in last section, the signal of interest is $ pp\rightarrow {\tiny W' W'} $, where in $\%64$ of probability, each $ W'$ decays hadronically to a tau and neutrino ($ W' \rightarrow \tau_{h} \nu $) and in $\%37$ of cases it decays leptonically to a meun or electron and a neutrino ($ W' \rightarrow \ell(e\mu) \nu  $). we calculated cross section of signal by considering different cuts in generation process and determined measure of increasing or suppressing of couplings after applying cuts. The generation implemented with 10000 events and at a center-of-mass energy of $ \sqrt{s}=8 $ TeV in each run. we also considered purely left-handed coupling  for $ g_L $ and $ g_R $ values.  The right-handed (left-handed) $W'$-boson couplings to quarks and leptons are considered to be $g'_R = 0$ and $g'_L = g_{(SM)}$ respectively, in other words decays to quarks and leptons are both included. A variety masses of $W'$ in range of 100 through 400 GeV were used for measuring cross sections.  Decay width or life time which is depends on decay modes or channels has a crucial role in conservation laws for appropriate quantum numbers, coupling strength of the decay process, and kinematic constraints , so decay widths corresponding to each mass of $ W' $ estimated concurrently by {\small MADGRAPH 5} and those calculated with this model compered to [ref] which investigated the LO and NLO partial and total width according to their branching fractions and cross sections for  $ W'_{R,L}\rightarrow t \bar{b},\bar{t}b $  and there was a good agreement among different values.  In table I we list the cross sections and total width for the decay of $ W' $ to both quarks and leptons in different mass. 

\begin{table}[htb]
	\centering
\begin{tabular}{|ccc|}
\hline 
Mass wprime(GeV)  &  Decay Width  &  Cross section(pb)\\
\hline 
100 & 2.50885 & 0.9696+-0.003  \\
130 & 3.26166& 0.3441+-0.0012 \\
160 & 4.014482 &0.1467+-0.0006 \\
190 & 4.837699 &0.06932+-0.00027 \\
 220& 5.881464& 0.03341+-0.00014 \\
 250 &6.988690 &0.0172+-6.01e-05 \\
 280 &8.103721 &0.00962+-3.8e-02 \\
 310 &9.210820 &0.00564+-2.4e-05 \\
 340 &10.30623 &0.003462+-1.3e-05 \\
 370 &11.39039& 0.002213+-8.9e-06\\ 
 400 &12.46427 &0.001451+-5.5e-06\\
\hline
\end{tabular}
\caption{Cross sections and decay widths corresponding to various $W'$ masses. \label{tab1} }
\end{table}

 

 
 In order to do a precise evaluation of cross sections sensibility under changes of coupling values, we estimated once half and again doubled left-handed coupling values. As can be seen in table 2 and 3, the amounts of cross section becamed 16 times (one-sixteenth) bigger (smaller) than cross sections in table I respectively.
  \begin{table}[htb]
	\centering
\begin{tabular}{|ccc|}
\hline 
Mass wprime(GeV)  &  Decay Width  &  Cross section(pb)\\
\hline 

 100& 0.62721& 0.06084+-0.00018\\
  130& 0.81542& 0.02156+-7e-05\\
  160& 1.00362& 0.009247+-3.7e-05\\
  190& 1.20944& 0.004351+-1.8e-05\\
  220& 1.47035& 0.00208+-8.3e-06\\
  250& 1.74718 &0.001077+-4.4e-06\\
  280& 2.02593& 0.0005993+-2.2e-06\\
  310& 2.30270& 0.000351+-1.3e-06\\
 340& 2.57655& 0.0002166+-8.3e-07\\
 370& 2.84757& 0.0001383+-5e-07\\
  400& 3.11605& 9.119e-05+-3.1e-07\\ 

\hline
\end{tabular}
\caption{Cross sections and decay widths in the case that $ g_R=0 , g_L=0.3241986  $ (half of the standard model) for various $W'$ masses. \label{tab1} }
\end{table}
  
 \begin{table}[htb]
	\centering
\begin{tabular}{|ccc|}
\hline 
Mass wprime(GeV)  &  Decay Width  &  Cross section(pb)\\
\hline 
100& 1.003553e+01& 15.08+-0.045\\
130 &1.304687e+01& 5.297+-0.017\\
160& 1.605799e+01& 2.26+-0.0072 \\
190&1.93512e+01& 1.064+-0.0025\\
220& 2.352564e+01& 0.5078+-0.0014\\
250& 2.795486e+01& 0.2619 ± 0.00072\\
280& 3.241489e+01& 0.1451 ± 0.00043\\
310& 3.684301e+01& 0.08521 ± 0.0002\\
340& 4.122483e+01& 0.05223 ± 0.00013\\
370& 4.556095e+01&0.03324 +- 9.314e-05\\
400& 4.985667e+01& 0.02192 ± 5.1e-05\\ 

\hline
\end{tabular}
\caption{Cross sections and decay widths in the case that $ g_R=0 , g_L=1.2967944  $ (twice of the standard model) for various $W'$ masses. \label{tab1} }
\end{table}

to be continued...
.
.
.
.
.
.




\section{Results}\label{sec:results} 

[Zamiri]

5 Interpretation and results

5.1 method and tools / ref to root

5.2 exclusion plots

5.2.1 Coupling SM, LEFT handed, exclusion vs. mass

5.2.2 Mass/(mixing angle) exclusion limit

5.3 quote the limit when the coupling is doubled


In this paper we used TAUOLA package~\cite{Davidson:2010rw,Jadach:1990mz,Golonka:2003xt,Jadach:1993hs} for simulation of our signal as well as root-v5.34.23 package \cite{Brun:1997pa}. TAUOLA is a library of Monte Carlo programs for leptonic and semileptonic decays of $\tau$-lepton. Fully information about final state consist of neutrinos, distribution of moderator particles and complete spin structure throughout the decay are some of the advantages of using this package. ROOT is a specialized software package that was put out at CERN to answer the purpose of particle physics community, especially for the data processing of the experimental set-ups located at the LHC. It is a $C^{++}$ library that contains a number of useful classes.  

 Calculation of efficiencies (total efficiency is defined by the fraction of events registered at the detector with respect to the number of events emitted by a radiation source) has been done for each channel: SR1, SR2, electron $\tau$ and meun $\tau$. It would be useful to report the amounts of efficiencies for different states of our left-handed generation, table \ref{eff-SM}  which expresses the values for  $ g_R=0 , g_L=0.6483972 $ (standard model) and table \ref{eff-half} , \ref{eff-twice} that express the values in the case that  $ g_R=0 , g_L=0.3241986 $ (half of the standard model) and $ g_R=0 , g_L=1.2967944 $ (twice of the standard model).

  \begin{table}[htb]
 	\centering
  	\begin{tabular}{|ccccc|}
  		\hline 
  		Mass Wprime & EfficiencySR1 & EfficiencySR2 & Efficiency $\mu$$\tau$ & Efficiency e $\tau$ \\
\hline 
  		100& 0.0458899\%& 0.614535\%& 0.011183\% & 0.00395297\%\\
 		130& 0.175685\%& 0.760096\%& 0.0442253\%& 0.0271803\%\\
  		160& 0.410097\%& 0.846774\%& 0.135277\%& 0.0756016\%\\
  		190& 0.715545\%& 0.909511\%& 0.228263\%& 0.189238\%\\
  		220& 1.06748\%& 0.968385\%& 0.43919\%& 0.331424\%\\
  		250& 1.47652\%& 0.983356\%& 0.632992\%& 0.516825\%\\ 
  		280& 1.80193\%& 1.02289\%& 0.861038\%& 0.703542\%\\
 		310& 2.14124\%& 1.05743\%& 1.12391\%& 0.865479\%\\
  		340& 2.45444\%& 1.03443\%& 1.46279\%& 1.18334\%\\ 
  		370& 2.76015\%& 1.00684\%& 1.6293\%& 1.38178\%\\
  		400& 2.99298\%& 0.97716\%& 1.87191\%& 1.62458\%\\

  		\hline
  	\end{tabular}
  	\caption{Efficiencies for $ g_R=0 , g_L=0.648 $ (standard model) corresponding to various W' masses. \label{eff-SM} }
  \end{table}


   \begin{table}[htb]
 	\centering
  	\begin{tabular}{|ccccc|}
  		\hline 
  		Mass Wprime  & EfficiencySR1  & EfficiencySR2 & Efficiency $\mu$$\tau$ & Efficiency e $\tau$ \\
                \hline 
100& 0.0466468\%& 0.614088\%& 0.0121252\%& 0.00439394\%\\
130& 0.168083\%& 0.787008\%& 0.0358624\%& 0.0263049\%\\
160& 0.425137\%& 0.866324\%& 0.109585\%& 0.0757335\%\\
190& 0.733011\%& 0.947878\%& 0.255279\%& 0.189645\%\\
220& 1.07375\%& 1.0056\%& 0.408889\%& 0.307338\%\\
250& 1.4634\%& 1.00851\%& 0.610881\%& 0.518507\%\\
280& 1.78294\%& 1.00314\%& 0.939803\%& 0.643615\%\\
310& 2.14762\%& 1.02447\%& 1.08377\%& 0.919447\%\\
340& 2.43738\%& 1.02806\%& 1.31608\%& 1.15041\%\\
370& 2.72227\%& 0.998014\%& 1.62246\%& 1.35562\%\\
400& 3.07908\%& 0.946486\%& 1.85765\%& 1.59617\%\\
  	
  	\hline
  	\end{tabular}
  	\caption{Efficiencies for $ g_R=0 , g_L=0.324 $ (half of the standard model) corresponding to various W' masses. \label{eff-half} }
  \end{table}
 

 \begin{table}[htb]
 	\centering
  	\begin{tabular}{|ccccc|}
  		\hline 
  		Mass Wprime  & EfficiencySR1  & EfficiencySR2 & Efficiency $\mu$$\tau$ & Efficiency e $\tau$ \\
 \hline 
100& 0.063603\%& 0.62375\%& 0.0166494\%& 0.00632205\%\\
130& 0.198096\%& 0.762918\%& 0.0462929\%& 0.0374592\%\\
160& 0.459054\%& 0.87826\%& 0.13438\%& 0.0971543\%\\
190& 0.76574\%& 0.961177\%& 0.275422\%& 0.197196\%\\
220& 1.09848\%& 0.971835\%& 0.443421\%& 0.363646\%\\
250& 1.48481\%& 0.976986\%& 0.623507\%& 0.522248\%\\
280& 1.79262\%& 0.994131\%& 0.870993\%& 0.742251\%\\
310& 2.16712\%& 0.99002\%& 1.09347\%& 0.892889\%\\
340& 2.41482\%& 1.0063\%& 1.35593\%& 1.11143\%\\
370& 2.72269\%&0.969054\%& 1.63439\%& 1.35672\%\\
400& 2.98607\%& 0.961773\%& 1.86216\%& 1.60924\%\\
  	
  	\hline
  	\end{tabular}
  	\caption{Efficiencies for $ g_R=0 , g_L=1.297 $ (twice of the standard model) corresponding to various W' masses. \label{eff-twice} }
  \end{table}
  

   

Almost every Particle Physics analysis uses some technique for separating signal from background. One type of approach that can be made towards claiming a discovery is "Hypothesis Testing". Hypothesis Testing is normally applied at the end-result of the experiment to test the hypothesis by establishing a confidence range for a particular variable at some specifed level. Upper limits based on the CLs method~\cite{0954-3899-28-10-313,Mistlberger:2012rs} were used in numerous publications of experimental results obtained at particle accelerator experiments such as LEP, the Tevatron and the LHC, most notable in the searches for new particles. 

%Discovery and exclusion potentials, false exclusion rates, etc that form the frequentist-motivated CLs method are derived from the probability density functions (pdfs) of $−2ln(Q)$, where $Q=L(s+b)/L(b)$ is the ratio of likelihoods for the two hypotheses of interest for the exclusion and discovery tests.

In this paper, the signal measured by using cross sections and efficiencies obtained from 8 TeV run of LHC and with 18.1 $fb^{-1}$ and 19.6 $fb^{-1}$ of integrated luminosity for purely hadronic channel and $\tau$-lepton channel. We used CLs method, an approximate confidence in the signal+background hypothesis for excluding signal points in confidence level with probability of 95\%. Exclusion at 95\% confidence level or CLs<0.05 Means that the probability to observe more events than seen in the data with the signal+background hypothesis (normalized to the probability in the background hypothesis only) is less than 5\%. Data yields and background preditions with uncertainties in the four signal regions of search obtained from~\cite{Khachatryan:2016trj} and have shown in \ref{tab:yields}. The symetric uncertainties for signal assumed to be 20\% for \lepTau channel and 25\% for \tauTau channel. 

\begin{table}[htb]
	\centering
	\begin{tabular}{|c|c|c|c|c|}
		\hline 
		\wprime  &e\Tau &  $\mu\Tau$ & \tauTau SR1 & \tauTau SR2 \\
		\hline 
	         Background &3.52&8.59&1.58&7.07\\     
		 Backgrounds Error& 3.39&4.83&0.65&2.25\\
		 Observed& 3&5&1&2\\
		
		\hline
	\end{tabular}
	\caption{Data yields and background predictions with errors for \lepTau and \tauTau channels. \label{tab:yields} }
\end{table}
 
In particle physic, sigma is used to represent standard deviation. Standard deviation measures the distribution of data points around a mean, or average, and can be thought of as how "wide" the distribution of points or values is. A sample with a high standard deviation is more spread out—it has more variability, and a sample with a low standard deviation clusters more tightly around the mean. The graph \ref{mixtwice} below shows the the upper limit of $\sigma/\sigma_{pp \rightarrow \wprime \wprime}$ production in $\pm\sigma$ when the coupling constant were doubled. As an individual can see in this graph the excluded mass of observed upper limit is $\wprime=320GeV$.
 We have done all this process for the two other situation that are mentioned before, when the coupling constants were halfed and also doubled. For the latter, no point is excluded, but for former the excluded observed mass was 360GeV.
for the mixture angles, in the case that $g_R=30^\circ,g_L=60^\circ$ and its complement the excluded mass of \wprime were about 310 GeV, however we measured ?? in the case that $g_R=45^\circ,g_L=45^\circ$. the graph {\bf ref} displays the exclution limit of mixture angles at different masses of \wprime.


\begin{figure}[!ht]
\centering
%\subfigure[{MET-hadronic channel}]{\label{mix30b}
\includegraphics*[width=.45\textwidth]{figs/mixtwice.pdf}
\hspace{3mm}
%\subfigure[{MET-$\tau$ lepton channel}]{\label{met-lh}
\includegraphics*[width=.45\textwidth]{figs/mixtwiceb.pdf}
\caption{Upper limit of $\sigma/\sigma_{pp \rightarrow \wprime \wprime}$ production for doubled coupling constant}
\label{fig:mixtwice}
\end{figure}


for the mixture angles,we have considered four mixture angles $0^\circ, 30^\circ, 45^\circ$ and $60^\circ$ for measuring cross section, efficieny and signal. for mixture angle of $90^\circ$ there is not any interaction with quarks, in other words it is compeletly right-handed. The efficiencies have been calculated for  $30^\circ, 45^\circ$ and $60^\circ$ have shown in tables \ref{eff-mix30} , \ref{eff-mix45} and \ref{eff-mix60} respectively.

 \begin{table}[htb]
 	\centering
  	\begin{tabular}{|ccccc|}
  		\hline 
  		Mass Wprime & EfficiencySR1 & EfficiencySR2 & Efficiency $\mu$$\tau$ & Efficiency e $\tau$ \\
\hline 
  		100&0.0425878\%&0.463907\%&0.0114524\%&0.00445605\%\\
        	130& 0.15833\%& 0.611521\%& 0.0215278\%& 0.0225592\%\\
        	160&0.355661\%&0.770315\%&0.119383\%&0.0712268\%\\
          	190& 0.675233\%& 0.823996\%& 0.253244\%& 0.170344\%\\
                220&0.98795\%&0.877546\%&0.367874\%&0.309155\%\\
                250&1.34316\%&0.915831\%&0.551174\%&0.441335\%\\ 
          	280& 1.69921\%& 1.919102\%& 0.798076\%& 0.682601\%\\
                310&2.04941\%&0.942801\%&1.08213\%&0.930369\%\\
                340& 2.3094\%& 0.974089\%& 1.32744\%& 1.05632\%\\ 
                370& 2.60532\%& 0.945003\%& 1.50113\%& 1.2786\%\\
        	400& 2.86564\%&0.940598\%&1.7122\%&1.48529\%\\

  		\hline
  	\end{tabular}
  	\caption{Efficiencies for  $ g_R=0.325, g_L=0.561  $ (mixture angle $30^\circ$) corresponding to various W' masses. \label{eff-mix30} }
  \end{table}

 \begin{table}[htb]
 	\centering
  	\begin{tabular}{|ccccc|}
  		\hline 
  		Mass Wprime & EfficiencySR1 & EfficiencySR2 & Efficiency $\mu$$\tau$ & Efficiency e $\tau$ \\
\hline 
  		100&0.0394144\%& 0.49005\%&0.0111105\%&0.00467822\%\\
        	130& 0.14495\%& 0.67301\%& 0.043722\%& 0.0244111\%\\
        	160& 0.387744\%&0.745843\%&0.091391\%&0.0742906\%\\
          	190& 0.670074\%& 0.841423\%& 0.191574\%& 0.166053\%\\
                220&1.03529\%&0.8487\%&0.416216\%&0.284929\%\\
                250&1.32378\%& 0.916378\%&0.560148\%&0.470159\%\\ 
          	280& 1.70549\%& 0.951027\%& 0.7956561\%& 0.632669\%\\
                310&1.98791\%&0.949235\%&1.02274\%&0.845121\%\\
                340& 2.39183\%& 0.954737\%& 1.29175\%& 1.13349\%\\ 
                370& 2.69404\%& 0.942885\%& 1.51814\%& 1.26643\%\\
        	400& 2.92952\%&0.972395\%&1.86994\%&1.52314\%\\

  		\hline
  	\end{tabular}
  	\caption{Efficiencies for $g_R=0.458,g_L=0.458$ (mixture angle $45^\circ$) corresponding to various W' masses. \label{eff-mix45} }
  \end{table}



 \begin{table}[htb]
 	\centering
  	\begin{tabular}{|ccccc|}
  		\hline 
  		Mass Wprime & EfficiencySR1 & EfficiencySR2 & Efficiency $\mu$$\tau$ & Efficiency e $\tau$ \\
\hline 
  		100&0.0360041\%&0.473165\%&0.00910814\%&0.00529326\%\\
        	130& 0.144957\%& 0.656349\%&0.0272752\%& 0.0228136\%\\
        	160&0.377702\%&0.737037\%&0.100728\%&0.0701156\%\\
          	190& 0.667001\%& 0.816705\%& 0.194563\%& 0.170054\%\\
                220&0.985845\%&0.892326\%&0.337702\%&0.288647\%\\
                250&1.3342\%&0.926464\%&0.577441\%&0.480671\%\\ 
          	280& 1.73498\%& 0.950296\%& 0.794418\%& 0.660901\%\\
                310&1.98017\%&1.01826\%&0.998056\%&0.83799\%\\
                340& 2.36194\%& 0.942754\%& 1.29175\%& 1.09907\%\\ 
                370& 2.64927\%& 0.988568\%& 1.51512\%& 1.33299\%\\
        	400&2.83053\%&0.988953\%&1.91958\%&1.58432\%\\

  		\hline
  	\end{tabular}
  	\caption{Efficiencies for  $ g_R=0.561 , g_L=0.325$  (mixture angle $60^\circ$) corresponding to various W' masses. \label{eff-mix60} }
  \end{table}

notice that the efficiencies of mixture angle $90^\circ$ represented in table \ref{eff-SM}. 


  The excluded mass of \wprime have masured for different mixture angles and have shown in table \ref{tab:ex mass} . the graphs \ref{fig:mix0},\ref{fig:mix30},\ref{fig:mix45} and \ref{fig:mix60} display the exclusion limit of mixture angles at different masses of \wprime. 

\begin{table}[htb]
	\centering
	\begin{tabular}{|c|c|}
		\hline 
		\wprime Mixture angle & excluded mass \\
		\hline 
		$0^\circ$ & 320\\
		$30^\circ$ & 225\\
		$45^\circ$ & 270\\
		$60^\circ$& ???\\
		\hline
	\end{tabular}
	\caption{Excluded masses of \wprime for different \wprime \label{tab:ex mass} }
\end{table}

%\begin{figure}[!ht]
%\centering
%\includegraphics*[width=.45\textwidth]{figs/brazilian flag.pdf}
%\hspace{3mm}
%\subfigure[{MET-$\tau$ lepton channel}]{\label{met-lh}
%\includegraphics*[width=.45\textwidth]{figs/MET-lh.pdf}
%\caption{Upper limit of $\sigma/\sigma_{pp \rightarrow \wprime\wprime}$ production }
%\label{fig:brazil}
%\end{figure}

 
\begin{figure}[!ht]
\centering
%\subfigure[{MET-hadronic channel}]{\label{mix30b}
\includegraphics*[width=.45\textwidth]{figs/mix0b.pdf}
\hspace{3mm}
%\subfigure[{MET-$\tau$ lepton channel}]{\label{met-lh}
\includegraphics*[width=.45\textwidth]{figs/mix0.pdf}
\caption{Upper limit of $\sigma/\sigma_{pp \rightarrow \wprime \wprime}$ production for mixangle $0^\circ$ }
\label{fig:mix0}
\end{figure}
 
\begin{figure}[!ht]
\centering
%\subfigure[{MET-hadronic channel}]{\label{mix30b}
\includegraphics*[width=.45\textwidth]{figs/mix30b.pdf}
\hspace{3mm}
%\subfigure[{MET-$\tau$ lepton channel}]{\label{met-lh}
\includegraphics*[width=.45\textwidth]{figs/mix30.pdf}
\caption{Upper limit of $\sigma/\sigma_{pp \rightarrow \wprime \wprime}$ production for mix angle $30^\circ$ }
\label{fig:mix30}
\end{figure}

\begin{figure}[!ht]
\centering
%\subfigure[{MET-hadronic channel}]{\label{mix30b}
\includegraphics*[width=.45\textwidth]{figs/mix45b.pdf}
\hspace{3mm}
%\subfigure[{MET-$\tau$ lepton channel}]{\label{met-lh}
\includegraphics*[width=.45\textwidth]{figs/mix45.pdf}
\caption{Upper limit of $\sigma/\sigma_{pp \rightarrow \wprime \wprime}$ production for mixangle $45^\circ$ }
\label{fig:mix45}
\end{figure}

\begin{figure}[!ht]
\centering
%\subfigure[{MET-hadronic channel}]{\label{mix30b}
\includegraphics*[width=.45\textwidth]{figs/mix60b.pdf}
\hspace{3mm}
%\subfigure[{MET-$\tau$ lepton channel}]{\label{met-lh}
\includegraphics*[width=.45\textwidth]{figs/mix60.pdf}
\caption{Upper limit of $\sigma/\sigma_{pp \rightarrow \wprime \wprime}$ production for mixangle $60^\circ$ }
\label{fig:mix60}
\end{figure}
 
In particle physic, sigma is used to represent standard deviation. Standard deviation measures the distribution of data points around a mean, or average, and can be thought of as how "wide" the distribution of points or values is. A sample with a high standard deviation is more spread out—it has more variability, and a sample with a low standard deviation clusters more tightly around the mean. The graph\ref{fig:brazil} below shows the the upper limit of $\sigma/\sigma_{pp \rightarrow \wprime \wprime}$ production in $\pm\sigma$. 
As an individual can see in this graph the excluded mass of observed upper limit is $\wprime=320GeV$.
 We have done all this process for the two other situation that are mentioned before, when the coupling constants were halfed and also doubled. For the latter, no point is excluded, but for former the excluded observed mass was 360GeV.
for the mixture angles, in the case that $g_R=30^\circ,g_L=60^\circ$ and its complement the excluded mass of \wprime were about 310 GeV, however we measured ?? in the case that $g_R=45^\circ,g_L=45^\circ$. the graph {\bf ref} displays the exclution limit of mixture angles at different masses of \wprime.

\begin{figure}[!ht]
\centering
\includegraphics*[width=.45\textwidth]{figs/brazilianFlag.pdf}
%\hspace{3mm}
%\subfigure[{MET-$\tau$ lepton channel}]{\label{met-lh}
%\includegraphics*[width=.45\textwidth]{figs/MET-lh.pdf}
\caption{Upper limit of $\sigma/\sigma_{pp \rightarrow \wprime \wprime}$ production }
\label{fig:brazil}
\end{figure}


\section{Conclusions}\label{sec:conclusion} 

6 Conclusion and discussion


\section{Acknowledgments}
Thank IPM.

\bibliography{WPrimePairBib}
%\begin{thebibliography}{99}

%\end{thebibliography}

\end{document}
