\section{Results}\label{sec:results} 
Using the \wprime samples generated by MadGraph as explained in Section \ref{sec:simulation} and decaying $\tau$ leptons using the TAUOLA package, we are ready to measure the efficiency of the selection for different channels for different \wprime masses. 

For each event, the probability of passing the selection cuts for a given signal region can be obtained using the cut efficiency tables of the experimental paper \cite{Khachatryan:2016trj}. In that paper, the efficiency of applying each cut on the reconstructed properties of the event is reported as a function of the generator level value of that property. It makes it very easy and accurate to take into account the detector effects that are always difficult to model. According to that paper, we can consider all the cuts independent and multiply the efficiency of different cuts to obtain the full selection efficiency for different channels and signal regions.

This was done for different \wprime masses and for different coupling strengths. The resulting efficiencies for the SM-like scenario can be seen in figure~\ref{fig:EfficiencyGraphs}. 
\begin{figure}[!htb]
	\centering
		\includegraphics*[width=.45\textwidth]{figs/EfficiencyGraphs.pdf}
	\caption{Efficiecny of signal selection in different signal regions as a function of the \wprime mass.}
	\label{fig:EfficiencyGraphs}
\end{figure}
The efficiencies for the case where \gL is increased or decreased by 50\% or where there is also a non-zero \gR are produced and compared with the results in 
figure~\ref{fig:EfficiencyGraphs}. As it is expected, the efficiencies depend only on the kinematic of the generated events which vary with the mass of the \wprime boson and do not depend on the coupling constants.
 
Having the full selection efficiency in one channel ($\varepsilon$), together with the production cross section ($\sigma$) and the decay branching ratio (BR), one can estimate the total number of expected signal events in a given integrated luminosity ($\mathcal{L}$) using the formula:
\begin{equation}
N^{channel}_{exp.}= \mathcal{L} \times \sigma(pp \to \wprime\wprime) \times BR^{2}(\wprime \to \tau \nu) \times \varepsilon
\end{equation}
According to the experimental paper, the integrated luminosity for the \tauTau ~signal regions is 18.1~fb$^{-1}$ and for the \lepTau ~channels is 19.6~fb$^{-1}$. Following the same reference, a systematic uncertainty of 20\% for \lepTau ~channel and 25\% for \tauTau ~channel is assumed. Data yields and background predictions with their uncertainties in the four signal regions of search obtained from Ref.\cite{Khachatryan:2016trj} and shown in table \ref{tab:yields}. 
\begin{table}[htb]
  \centering
  \caption{Data yields and background predictions with their uncertainties for \lepTau ~and \tauTau ~channels. The shown uncertainty is the quadratic sum of the statistical and systematic uncertainties provided by the CMS experiment.\label{tab:yields} }
  \begin{tabular}{|c|c|c|c|c|}
    \hline 
               &    e\Tau       &  $\mu\Tau$     &  \tauTau ~SR1  & \tauTau ~SR2 \\
    \hline 
    Background &3.52 $\pm$ 3.39 &8.59 $\pm$ 4.83 &1.58 $\pm$0.65 &7.07 $\pm2.25$ \\     
    Observed data& 3            &      5         &    1          &    2    \\  
    \hline
  \end{tabular}
\end{table}

The 95\% confidence level upper limit on the signal strength can be found by combining all the four channels. A Likelihood ratio semi-bayesian method implemented in ROOT \cite{Brun:1997pa} is used. Signal strength is defined as the $\sigma /\sigma_{pp \to \wprime\wprime}$ratio. Results for the SM-like \wprime are shown in figure  \ref{fig:brazilianFlags} (top-left). 
\begin{figure}[!htb]
  \centering
  \includegraphics*[width=.45\textwidth]{figs/mix0b.pdf}
  \vspace{3mm}	
  \includegraphics*[width=.45\textwidth]{figs/mix30b.pdf}
  \hspace{3mm}
  \includegraphics*[width=.45\textwidth]{figs/mix45b.pdf}
  \vspace{3mm}	
  \includegraphics*[width=.45\textwidth]{figs/mix60b.pdf}
  \hspace{3mm}
  \includegraphics*[width=.45\textwidth]{figs/mixHalvedb.pdf}
  \vspace{3mm}
  \includegraphics*[width=.45\textwidth]{figs/mix3Over2b.pdf}
  \caption{Upper limit of $\sigma/\sigma_{pp \rightarrow \wprime \wprime}$ production for different scenarios}
  \label{fig:brazilianFlags}
\end{figure}
It can be seen that \wprime masses up to the 290 GeV are excluded.  

Repeating this procedure for the other scenarios of the coupling constants, it is observed that when \gL ~is decreased, the sensitivity is decreased, and vice versa, as it is expected. Figure \ref{fig:brazilianFlags} shows the observed limits, the expected exclusions and $\pm 1 ~\sigma$ uncertainty on the expected exclusions for different scenarios of the coupling constants. Table \ref{tab:ObservedLimits}
\begin{table}[htb]
	\centering
	\caption{The expected and observed lower limits on the \wprime mass in different scenarios of the coupling constants. \label{tab:ObservedLimits} }
	\begin{tabular}{|c|c|c|}
	\hline
	\gR, \gL           & Expected (GeV) & Observed (GeV) \\\hline
    0, 0.64            &   255    &    290  \\
	0.32, 0.56         &   190    &    225  \\
	0.46, 0.46         &   135    &    170  \\
	0.56, 0.32         &   90     &    120  \\
    0, 0.32            &   90     &    120  \\
    0, 0.96            &   380    &    420  \\\hline
	\end{tabular}
\end{table}
summarizes the expected and observed limits in different scenarios. Always, the observed limit is higher than the expected one, because in different signal regions the observed data is less than the expected background (Table. \ref{tab:yields}).

The results are lower than the results from the direct search, but the proposed method can be used to constrain any new model with a similar final state, without need to simulate the response of a real detector.
