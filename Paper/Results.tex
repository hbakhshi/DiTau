\section{Results}\label{sec:results} 

%[Zamiri]
%5 Interpretation and results
%5.1 method and tools / ref to root
%5.2 exclusion plots
%5.2.1 Coupling SM, LEFT handed, exclusion vs. mass
%5.2.2 Mass/(mixing angle) exclusion limit
%5.3 quote the limit when the coupling is doubled
% root-v5.34.23 package 

In this paper the TAUOLA package~\cite{Davidson:2010rw} is used to simulate the $\tau$ lepton decays. It simulates the hadronic and leptonic decays of the $\tau$ lepton and provides full information about final state particles including neutrinos and mediator particles. It also considers spin information of the decay products in simulating the angular distribution of the decay products.

Using the \wprime samples generated by MadGraph as explained in section \ref{sec:simulation} and decaying $\tau$ leptons using the TAUOLA package, we are ready to measure the efficiency of the selection for different channels for different \wprime masses. For this purpose, the visible momentum of the hadronic decaying $\tau$ is defined as the original $\tau$ momentum before decay subtracted by the momentum of the Neutrinos in the decay chain. The missing transverse momentum (\MET) of the event is also defined as the negative of the sum of the visible \pt ~of the two $\tau$ leptons. Having these information, one can construct all the needed variables like the transverse mass of the leptons or \mttwo. 

For each event, the probability of passing the selection cuts for a given signal region can be obtained using the cut efficiency tables of the experimental paper \cite{Khachatryan:2016trj}. In that paper, the efficiency of applying each cut on the reconstructed properties of the event is reported as a function of the generator level value of that property. It makes it very easy and accurate to take into account the detector effects that are always difficult to model. As claimed in that paper, we can consider all the cuts independent and multiply the efficiency of different cuts to obtain the full selection efficiency for different channels and signal regions.

This was done for different \wprime masses and for different coupling strengths. You can see the resulting efficiencies for the SM like scenario in table ref{eff-SM}. The results for the case where coupling is two times or half of the standard-model are reported in tables ref{eff-twice} and ref{eff-half} respectively.



Selection efficiencies are also calculated for different mixing angles. For that we have considered four mixing angles : $0^\circ, 30^\circ, 45^\circ$ and $60^\circ$. The mixing angle of $90^\circ$ is not considered, because in that case there is not any interaction with leptons. The mixing angle of $0^\circ$ is the fully left-handed scenario, for which efficiencies are reported in table~ref{eff-SM}.  The selection efficiencies for  $30^\circ, 45^\circ$ and $60^\circ$ have been reported in tables ref{eff-mix30} , ref{eff-mix45} and ref{eff-mix60} respectively.



 
Having the selction efficiency in one channel together with the cross section and branching fraction we can estimate the total number of expected signal events in a given luminosity using the following formula :
\begin{equation}
N^{channel}_{exp.}= \mathcal{L} \times \sigma(pp \to \wprime\wprime) \times BR^{2}(\wprime \to \tau \nu) \times \varepsilon^{channel}_{full~selection}
\end{equation}
As mentioned in the experimental paper, the luminosity for the \tauTau ~signal regions is 18.1~fb$^{-1}$ and for the \lepTau channels is 19.6~fb$^{-1}$. A symmetric uncertainty of 20\% for \lepTau channel and 25\% for \tauTau channel is assumed. Data yields and background preditions with uncertainties in the four signal regions of search obtained from Ref.\cite{Khachatryan:2016trj} and have shown in table \ref{tab:yields}. 

\begin{table}[htb]
  \centering
  \begin{tabular}{|c|c|c|c|c|}
    \hline 
    &e\Tau &  $\mu\Tau$ & \tauTau SR1 & \tauTau SR2 \\
    \hline 
    Background &3.52 $\pm$ 3.39 &8.59 $\pm$ 4.83 &1.58 $\pm$0.65 &7.07 $\pm2.25$ \\     
    Observed data& 3&5&1&2\\  
    \hline
  \end{tabular}
  \caption{Data yields and background predictions with their uncertainties for \lepTau ~and \tauTau ~channels. \label{tab:yields} }
\end{table}

The 95\% confidence level upper limit on the signal strength can be found by combining all the four channels. A Likelihood ratio semi-bayesian method implemented in ROOT \cite{Brun:1997pa} is used. Signal strength is defined as $r = \frac{ \sigma }{\sigma_{pp \to \wprime\wprime} }$. Results for the standard model like \wprime are shown in figure  \ref{fig:brazilianFlags}. It can be seen that \wprime masses up to the $320GeV$ are excluded.  Repeating this procedure for the two other scenarios for the coupling, we observe that when the coupling constants are halved, the sensitivity is very low and no point is excluded, but for the \wprime with a doubled coupling masses up to $360GeV$ are excluded (See figure \ref{fig:brazilianFlags}).
\begin{figure}[!htb]
  \centering
  \includegraphics*[width=.45\textwidth]{figs/mix0b.pdf}
  \vspace{3mm}	
  \includegraphics*[width=.45\textwidth]{figs/mix30b.pdf}
  \hspace{3mm}
  \includegraphics*[width=.45\textwidth]{figs/mix45b.pdf}
  \vspace{3mm}	
  \includegraphics*[width=.45\textwidth]{figs/mix60b.pdf}
  \hspace{3mm}
  \includegraphics*[width=.45\textwidth]{figs/mixHalvedb.pdf}
  \hspace{3mm}
  \includegraphics*[width=.45\textwidth]{figs/mix3Over2b.pdf}
  \caption{Upper limit of $\sigma/\sigma_{pp \rightarrow \wprime \wprime}$ production for different scenarios}
  \label{fig:brazilianFlags}
\end{figure}

When the mixing angle between the right and left-handed \wprime is varied, in the case of mixing angle $\theta=30^\circ$ and $\theta=60^\circ$ masses up to 225 GeV are excluded, however for the case of $\theta=45^\circ$ the limit on the mass is more sensitive and goes up to 270 GeV. The results are shown separately in figures~\ref{fig:brazilianFlags}. The graph ~\ref{fig:brazilianFlags} displays the exclusion limit of mixture angles at different masses of \wprime.

 
