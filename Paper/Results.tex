\section{Results}\label{sec:results} 

[Zamiri]

5 Interpretation and results

5.1 method and tools / ref to root

5.2 exclusion plots

5.2.1 Coupling SM, LEFT handed, exclusion vs. mass

5.2.2 Mass/(mixing angle) exclusion limit

5.3 quote the limit when the coupling is doubled


In this paper we used TAUOLA package~\cite{Davidson:2010rw,Jadach:1990mz,Golonka:2003xt,Jadach:1993hs} for simulation of our signal as well as root-v5.34.23 package \cite{Brun:1997pa}. TAUOLA is a library of Monte Carlo programs for leptonic and semileptonic decays of $\tau$-lepton. Fully information about final state consist of neutrinos, distribution of moderator particles and complete spin structure throughout the decay are some of the advantages of using this package. ROOT is a specialized software package that was put out at CERN to answer the purpose of particle physics community, especially for the data processing of the experimental set-ups located at the LHC. It is a $C^{++}$ library that contains a number of useful classes.  

 Calculation of efficiencies (total efficiency is defined by the fraction of events registered at the detector with respect to the number of events emitted by a radiation source) has been done for each channel: SR1, SR2, electron $\tau$ and meun $\tau$. It would be useful to report the amounts of efficiencies for different states of our left-handed generation, table \ref{eff-SM}  which expresses the values for  $ g_R=0 , g_L=0.6483972 $ (standard model) and table \ref{eff-half} , \ref{eff-twice} that express the values in the case that  $ g_R=0 , g_L=0.3241986 $ (half of the standard model) and $ g_R=0 , g_L=1.2967944 $ (twice of the standard model).

  \begin{table}[htb]
 	\centering
  	\begin{tabular}{|ccccc|}
  		\hline 
  		Mass Wprime & EfficiencySR1 & EfficiencySR2 & Efficiency $\mu$$\tau$ & Efficiency e $\tau$ \\
\hline 
  		100& 0.0458899\%& 0.614535\%& 0.011183\% & 0.00395297\%\\
 		130& 0.175685\%& 0.760096\%& 0.0442253\%& 0.0271803\%\\
  		160& 0.410097\%& 0.846774\%& 0.135277\%& 0.0756016\%\\
  		190& 0.715545\%& 0.909511\%& 0.228263\%& 0.189238\%\\
  		220& 1.06748\%& 0.968385\%& 0.43919\%& 0.331424\%\\
  		250& 1.47652\%& 0.983356\%& 0.632992\%& 0.516825\%\\ 
  		280& 1.80193\%& 1.02289\%& 0.861038\%& 0.703542\%\\
 		310& 2.14124\%& 1.05743\%& 1.12391\%& 0.865479\%\\
  		340& 2.45444\%& 1.03443\%& 1.46279\%& 1.18334\%\\ 
  		370& 2.76015\%& 1.00684\%& 1.6293\%& 1.38178\%\\
  		400& 2.99298\%& 0.97716\%& 1.87191\%& 1.62458\%\\

  		\hline
  	\end{tabular}
  	\caption{Efficiencies for $ g_R=0 , g_L=0.6483972 $ (standard model) corresponding to various W' masses. \label{eff-SM} }
  \end{table}


   \begin{table}[htb]
 	\centering
  	\begin{tabular}{|ccccc|}
  		\hline 
  		Mass Wprime  & EfficiencySR1  & EfficiencySR2 & Efficiency $\mu$$\tau$ & Efficiency e $\tau$ \\
                \hline 
100& 0.0466468\%& 0.614088\%& 0.0121252\%& 0.00439394\%\\
130& 0.168083\%& 0.787008\%& 0.0358624\%& 0.0263049\%\\
160& 0.425137\%& 0.866324\%& 0.109585\%& 0.0757335\%\\
190& 0.733011\%& 0.947878\%& 0.255279\%& 0.189645\%\\
220& 1.07375\%& 1.0056\%& 0.408889\%& 0.307338\%\\
250& 1.4634\%& 1.00851\%& 0.610881\%& 0.518507\%\\
280& 1.78294\%& 1.00314\%& 0.939803\%& 0.643615\%\\
310& 2.14762\%& 1.02447\%& 1.08377\%& 0.919447\%\\
340& 2.43738\%& 1.02806\%& 1.31608\%& 1.15041\%\\
370& 2.72227\%& 0.998014\%& 1.62246\%& 1.35562\%\\
400& 3.07908\%& 0.946486\%& 1.85765\%& 1.59617\%\\
  	
  	\hline
  	\end{tabular}
  	\caption{Efficiencies for $ g_R=0 , g_L=0.3241986 $ (half of the standard model) corresponding to various W' masses. \label{eff-half} }
  \end{table}
 

 \begin{table}[htb]
 	\centering
  	\begin{tabular}{|ccccc|}
  		\hline 
  		Mass Wprime  & EfficiencySR1  & EfficiencySR2 & Efficiency $\mu$$\tau$ & Efficiency e $\tau$ \\
 \hline 
100& 0.063603\%& 0.62375\%& 0.0166494\%& 0.00632205\%\\
130& 0.198096\%& 0.762918\%& 0.0462929\%& 0.0374592\%\\
160& 0.459054\%& 0.87826\%& 0.13438\%& 0.0971543\%\\
190& 0.76574\%& 0.961177\%& 0.275422\%& 0.197196\%\\
220& 1.09848\%& 0.971835\%& 0.443421\%& 0.363646\%\\
250& 1.48481\%& 0.976986\%& 0.623507\%& 0.522248\%\\
280& 1.79262\%& 0.994131\%& 0.870993\%& 0.742251\%\\
310& 2.16712\%& 0.99002\%& 1.09347\%& 0.892889\%\\
340& 2.41482\%& 1.0063\%& 1.35593\%& 1.11143\%\\
370& 2.72269\%&0.969054\%& 1.63439\%& 1.35672\%\\
400& 2.98607\%& 0.961773\%& 1.86216\%& 1.60924\%\\
  	
  	\hline
  	\end{tabular}
  	\caption{Efficiencies for $ g_R=0 , g_L=1.2967944 $ (twice of the standard model) corresponding to various W' masses. \label{eff-twice} }
  \end{table}
  

   

Almost every Particle Physics analysis uses some technique for separating signal from background. One type of approach that can be made towards claiming a discovery is "Hypothesis Testing". Hypothesis Testing is normally applied at the end-result of the experiment to test the hypothesis by establishing a confidence range for a particular variable at some specifed level. Upper limits based on the CLs method~\cite{0954-3899-28-10-313,Mistlberger:2012rs} were used in numerous publications of experimental results obtained at particle accelerator experiments such as LEP, the Tevatron and the LHC, most notable in the searches for new particles. 

%Discovery and exclusion potentials, false exclusion rates, etc that form the frequentist-motivated CLs method are derived from the probability density functions (pdfs) of $−2ln(Q)$, where $Q=L(s+b)/L(b)$ is the ratio of likelihoods for the two hypotheses of interest for the exclusion and discovery tests.

In this paper, the signal measured by using cross sections and efficiencies obtained from 8 TeV run of LHC and with 18.1 $fb^{-1}$ and 19.6 $fb^{-1}$ of integrated luminosity for purely hadronic channel and $\tau$-lepton channel. We used CLs method, an approximate confidence in the signal+background hypothesis for excluding signal points in confidence level with probability of 95\%. Exclusion at 95\% confidence level or CLs<0.05 Means that the probability to observe more events than seen in the data with the signal+background hypothesis (normalized to the probability in the background hypothesis only) is less than 5\%. Data yields and background preditions with uncertainties in the four signal regions of search obtained from~\cite{Khachatryan:2016trj} and have shown in \ref{tab:yields}. The symetric uncertainties for signal assumed to be 20\% for \lepTau channel and 25\% for \tauTau channel. 

\begin{table}[htb]
	\centering
	\begin{tabular}{|c|c|c|c|c|}
		\hline 
		\wprime  &e\Tau &  $\mu\Tau$ & \tauTau SR1 & \tauTau SR2 \\
		\hline 
	         Background &3.52&8.59&1.58&7.07\\     
		 Backgrounds Error& 3.39&4.83&0.65&2.25\\
		 Observed& 3&5&1&2\\
		
		\hline
	\end{tabular}
	\caption{Data yields and background predictions with errors for \lepTau and \tauTau channels. \label{tab:yields} }
\end{table}
 
In particle physic, sigma is used to represent standard deviation. Standard deviation measures the distribution of data points around a mean, or average, and can be thought of as how "wide" the distribution of points or values is. A sample with a high standard deviation is more spread out—it has more variability, and a sample with a low standard deviation clusters more tightly around the mean. The graph\ref{fig:brazil} below shows the the upper limit of $\sigma/\sigma_{pp \rightarrow \wprime \wprime}$ production in $\pm\sigma$. 
As an individual can see in this graph the excluded mass of observed upper limit is $\wprime=320GeV$.
 We have done all this process for the two other situation that are mentioned before, when the coupling constants were halfed and also doubled. For the latter, no point is excluded, but for former the excluded observed mass was 360GeV.
for the mixture angles, in the case that $g_R=30^\circ,g_L=60^\circ$ and its complement the excluded mass of \wprime were about 310 GeV, however we measured ?? in the case that $g_R=45^\circ,g_L=45^\circ$. the graph {\bf ref} displays the exclution limit of mixture angles at different masses of \wprime.

\begin{figure}[!ht]
\centering
\includegraphics*[width=.45\textwidth]{figs/brazilianFlag.pdf}
%\hspace{3mm}
%\subfigure[{MET-$\tau$ lepton channel}]{\label{met-lh}
%\includegraphics*[width=.45\textwidth]{figs/MET-lh.pdf}
\caption{Upper limit of $\sigma/\sigma_{pp \rightarrow \wprime \wprime}$ production }
\label{fig:brazil}
\end{figure}
