\section{SIMULATION}\label{sec:evo}

{\bf The text should be updated to address the following items discussed in the meeting.}

4 [Zamiri] Simulation of signal events

4.1 madgraph

4.2 reference to the wprime package that used

4.3 quote coupling value

4.3.1 doubled and halfed

4.4 explain about the mixture of lef-and right handed

4.5 decay width and validation (ref to the W->tb)

4.6 cross sections (mass, mixture angle)

4.7 discussion about the branching fractions

4.7.1 assume SM values

4.7.2 assume that it prefers tau between leptons

4.8 validation plots ??? / kinematic distributions

4.8.1 gen mt2 for different masses

4.8.2 gen tau-pt for different masses

4.9 yields and final efficiencies

4.9.1 2 tables for SM coupling and double coupling (for different masses)

4.9.2 1 table for SM coupling and different mixing angles = (0,30,45,60) (for 
mass=220GeV)






As it mentioned in last section, the signal of intrest is $ pp\rightarrow W' $, where in $\%64$ of probability, W' decays hadronically to a tau and neutrino ($ W' \rightarrow \tau_{h} \nu $) and in $\%37$ of cases it decays leptonically to a meun or electron and a neutrino ($ W' \rightarrow \ell(e\mu) \nu  $). We used MadGraph \cite{Alwall:2011uj} for simulating signal and backgrands events. we calculated cross section of signal by considering different cuts in generation process and determined measure of increasing or suppressing of couplings after applying cuts. 
The cross section of signal calulated in various mass of W' and corresponding to every mass of W', decay widths estimated concurrently. Decay width or life time which is depends on decay modes or channels has a crucial role In table I we list the cross sections and total width for the decay of W' to both quarks and leptons in different mass.

 \begin{table}[htb]
	\centering
\begin{tabular}{|ccc|}
\hline 
Mass wprime  &  Decay Width  &  Cross section\\
\hline 
100 & 2.50885 & 0.9696+-0.003  \\
130 & 3.26166& 0.3441+-0.0012 \\
160 & 4.014482 &0.1467+-0.0006 \\
190 & 4.837699 &0.06932+-0.00027 \\
 220& 5.881464& 0.03341+-0.00014 \\
 250 &6.988690 &0.0172+-6.01e-05 \\
 280 &8.103721 &0.00962+-3.8e-02 \\
 310 &9.210820 &0.00564+-2.4e-05 \\
 340 &10.30623 &0.003462+-1.3e-05 \\
 370 &11.39039& 0.002213+-8.9e-06\\ 
 400 &12.46427 &0.001451+-5.5e-06\\
\hline
\end{tabular}
\caption{Cross sections and decay widths corresponding to various W' masses. \label{tab1} }
\end{table}

 
Befor explanation about details, definition of some variable and concepts which used for collecting our data should be represent. 