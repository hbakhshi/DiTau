\section{SIMULATION}\label{sec:simulation}

{\bf The text should be updated to address the following items discussed in the meeting.}

4 [Zamiri] Simulation of signal events 

4.1 madgraph {\bf Done}

4.2 reference to the wprime package that used {\bf Done}

4.3 quote coupling value {\bf Done}

4.3.1 doubled and halved {\bf Done}

4.4 explain about the mixture of left-and right handed {\bf Done}

4.5 decay width and validation (ref to the W->tb) {\bf Done}

4.6 cross sections (mass, mixture angle) {\bf Done}

4.7 discussion about the branching fractions {\bf Done??}

4.7.1 assume SM values {\bf Done}

4.7.2 assume that it prefers tau between leptons

4.8 validation plots ??? / kinematic distributions 

4.8.1 gen mt2 for different masses {\bf Done}

4.8.2 gen tau-pt for different masses {\bf Done}

4.9 yields and final efficiencies 

4.9.1 2 tables for SM coupling and double coupling (for different masses)
{\bf Done} 

4.9.2 1 table for SM coupling and different mixing angles = (0,30,45,60) (for mass=220GeV) {\bf Done}


%{\small MADGRAPH 5}[ref] is the new version of {\small MADGRAPH} matrix generator~\cite{Alwall:2011uj}. 
We used MadGraph5\_aMC@NLO~\cite{Alwall:2014hca}, which is the new version of MadGraph5~\cite{Alwall:2011uj} matrix generator, to generate signal events. In continue, we refer to this event generator it as MadGraph.
%beacause implementation of accurate simulation for signal is based on the ability of {\small MADGRAPH}. 
%It is the tool with stronge power of extracting physics from the data and making precise prediction for physics Beyond the Standard Model {\small (BSM)}. 
%Generating tree-level matrix elements and measurement of cross sections for classes of renormalizable models and establishment of {\small FEYNRULES}~\cite{Christensen:2008py} to perform new physics models and finaly making links between {\small FEYNRULES} and matrix elements and standardizing them by Universal F{\small EYNRULES} Output format, the {\small (UFO)}~\cite{Degrande:2011ua}, are some of the significant benefits of {\small MADGRAPH 5}. 
The model used for signal generation is $\wprime$ effective model (WEff-UFO)~\cite{Sullivan:2002jt}, that is an extension of SM by adding \wprime interactions with the SM particles.
The Universal FEYNRULES\cite{Christensen:2008py} Output format (UFO) is a flexible description of new models providing the abstract information about the particles, parameters and vertices of a new model that can be used by matrix-element generators. 
%One can also achieve to example of Mathematica notebook that loads the model, calculates the Feynman rules and extract the model files within the UFO format in [ref0]. 
%The \wprime Effective model authenticated the using MadGraph searches consist of $\wprime -> tb$ vertex at the LHC~\cite{ATLAS-CONF-2013-050}.
As it is mentioned in last section, the signal of interest is $ pp\rightarrow W^{\prime+} W^{\prime-}$. 
%We calculate cross section of signal by considering different parameters in generation process and determine measure of increasing or suppressing of cross section after changing parameters. 
For each set of parameters, 10000 events are generated at a center-of-mass energy of 8 TeV. We considered purely left-handed  \wprime ($ g_L = g_{SM}$ and $ g_R $ = 0). 
%The right-handed (left-handed) \wprime-boson couplings to quarks and leptons are considered to be $g'_R = 0$ and $g'_L = g_{(SM)}$ respectively, 
This means interactions with quarks and leptons are both allowed. Different masses of  \wprime boson in the range of 100 through 400 GeV are used in this analysis.  
Decay width or life time of \wprime boson depends on decay modes, coupling strength of the decay process, and kinematic constraints. Decay widths corresponding to each mass of \wprime boson are estimated  by MadGraph. The results agree with the values reported in Ref.\cite{Sullivan:2002jt} where the total width of \wprime and partial width of $\wprime \rightarrow t \bar{b},\bar{t}b $ are calculated in the leading order and the next to leading order precision.
The production cross section and total width for the decay of \wprime  to both quarks and leptons in different masses are listed in table \ref{tab:Xsec,L-h}. The values are for  proton-proton (pp) collisions at $\sqrt{s}$ = 8 TeV.
 \begin{table}[!htb]
	\centering
\begin{tabular}{|c|c|c|}
\hline 
\wprime Mass (GeV)  &  Decay Width (GeV)  &  Cross section (fb)\\
\hline 
100 & 2.509 & 970  \\
130 & 3.262 & 344 \\
160 & 4.014 &147 \\
190 & 4.838 &69.3 \\
 220& 5.881 & 33.4 \\
 250 &6.989 &17.2 \\
 280 &8.104 &9.62 \\
 310 &9.211 &5.64 \\
 340 &10.31 &3.46 \\
 370 &11.39 & 2.21\\ 
 400 &12.46 &1.45\\
\hline
\end{tabular}
\caption{Cross sections and decay widths in the case that $ g_R=0 , g_L= g_{SM} = 0.6483972 $ {\bf It was 0.458486057 in the original text. What about the code??? It should be rounded.} for various \wprime masses for pp collisions at $\sqrt{s}$ = 8 TeV. The relative uncertainties on the cross sections are less than 1\%. \label{tab:Xsec,L-h} }
\end{table}


As a cross check the branching fraction which is defined as the partial decay width to a special channel divided by the total width
It can be observed in tables \ref{tab:W'Plus} and \ref{tab:W'Minus} that the values are consistent for $W'^+$ and $W'^-$. In 8-9\% of the cases \wprime boson decays to $\tau$ lepton and the corresponding neutrino and in about 27\% of cases it decays to top and  bottom quark. Due to the fact that by choosing mass range of 200 to 400 (GeV), decay of $ \wprime^\pm $ boson to top quarks is kinematically allowed, the decay of the $ \wprime\rightarrow \tau,\nu_\tau $ is almost constant with increasing the \wprime mass.
{\bf I think we should report the values for a left-handed \wprime.
	
	The values in the tables should be rounded.
	
	The reported values for the situation that the mixture angle of $ g_R$ and  $g_L$ is assumed to be $45^\circ $. }
\begin{table}[htb]
	\centering
	\begin{tabular}{|c|c|c|}
		\hline 
		\wprime Mass (GeV)  &  BR $ W'^+\rightarrow \tau,\nu_\tau $& BR $ W'^+\rightarrow  t \bar{b},\bar{t}b $ \\
		\hline 
		200 & 0.082610\% & 0.03379\%\\
		400 &0.07665\%& 0.19082\%\\
		600 &0.07701\%&0.22118\%\\
		800&0.07799\% &0.23229\%\\
		\hline
	\end{tabular}
	\caption{Cross sections and branching fractions of $W'^+$ in the case that $ g_R=0.458486057 , g_L=0.458486057  $ for various $W'^+$ masses. \label{tab:W'Plus} }
\end{table}
\begin{table}[htb]
	\centering
	\begin{tabular}{|c|c|c|}
		\hline 
		\wprime Mass (GeV)  &  BR $ W'^-\rightarrow \tau,\nu_\tau $ & BR $ W'^-\rightarrow  t \bar{b},\bar{t}b $ \\
		\hline 
		200 &  0.08682\% &0.03368\% \\
		400  &0.07847\% &0.191023\% \\
		600  &0.07827\%&0.22194\%\\
		800  &0.079037\% &0.23348\% \\
		\hline
	\end{tabular}
	\caption{Cross sections and branching fractions of $W'^-$ in the case that $ g_R=0.458486057 , g_L=0.458486057  $ for various $W'^-$ masses. \label{tab:W'Minus} }
\end{table}


The couplings of \wprime are not fixed by the model, so to investigate the effect of the couplings, we calculate the production cross section and decay width when the couplings are multiplied by 2 or 0.5. As can be seen in tables \ref{tab:Xsec,half} and \ref{tab:Xsec,twice}, the cross section is scaled by 16 and 1/16 when the coupling is doubled and halved, respectively. It is noticeable that the values of decay widths are proportional to  factors of 4 and  1/4 for the doubled and halved left-handed coupling values.
\begin{table}[htb]
	\centering
\begin{tabular}{|c|c|c|}
\hline 
\wprime Mass (GeV)  &  Decay Width (GeV) &  Cross section(fb)\\
\hline 
  100& 0.627& 60.8\\
  130& 0.815& 21.6\\
  160& 1.004& 9.25\\
  190& 1.209& 4.35\\
  220& 1.470& 2.08\\
  250& 1.747 &1.08\\
  280& 2.026& 0.599\\
  310& 2.303& 0.351\\
  340& 2.577& 0.217\\
  370& 2.848& 0.138\\
  400& 3.116 & 0.0912\\ 
\hline
\end{tabular}
\caption{Cross sections and decay widths in the case that $ g_R=0 , g_L=0.3241986  $ (half of the standard model) for various \wprime masses. \label{tab:Xsec,half} }
\end{table}  
\begin{table}[htb]
	\centering
\begin{tabular}{|c|c|c|}
\hline 
\wprime Mass (GeV)  &  Decay Width (GeV) &  Cross section(fb)\\
\hline 
100& 10.04& 15100\\
130& 13.05& 5300\\
160& 16.06& 2260 \\
190& 19.35& 1060\\
220& 23.53& 508\\
250& 27.95& 262\\
280& 32.41& 145 \\
310& 36.84& 85.2 \\
340& 41.22& 52.2\\
370& 45.56& 33.2 \\
400& 49.86& 21.9 \\ 
\hline
\end{tabular}
\caption{Cross sections and decay widths in the case that $ g_R=0 , g_L=1.2967944  $ (twice of the standard model) for various \wprime masses. \label{tab:Xsec,twice} }
\end{table}




In an alternative approach, the left-handed and right-handed couplings are changed in a way that their squared sum is constant and equal to $g_{SM}^2$.
\begin{equation}
g_{SM}^2 = (g'_L)^2 +  (g'_R)^2 
\end{equation}
It is easier, to define a mixing angle and rewrite the couplings as:
\begin{eqnarray}
g'_L  = g_{SM} cos\theta \\
g'_R  = g_{SM} sin\theta
\end{eqnarray}
By varying $\theta$ from 0 to $ 90^\circ $, the \wprime goes from a purely left-handed to purely right-handed vector boson. The latter \wprime does not have any interaction with the leptons. 
The variation of cross sections and decay widths due to various mixing angles, for \wprime boson with a mass of 220 GeV is shown in table \ref{tab:mixingAngle}.
\begin{table}[htb]
	\centering
	\begin{tabular}{|c|c|c|c|}
		\hline 
		Mixing angle $\theta$  &  Decay Width (GeV) & BR(\wprime $\rightarrow \tau \nu_\tau$)  &  Cross section (fb)\\
		\hline 
		0      &          &          & \\
		30      &          &          & \\
		45      &          &          & \\
		60      &          &          & \\
		90      &          &          & \\
%		$g_R=45^\circ,g_L=45^\circ$& 4.997& 4.99$\pm$0.0053\\
%		$g_R=30^\circ,g_L=60^\circ$ &5.452& 5.452$\pm$0.0054\\
%		$g_R=60^\circ,g_L=30^\circ$& 4.532& 4.532$\pm$0.0053 \\
%		$g_R=0^\circ,g_L=90^\circ$&5.881& 5.881$\pm$0.0059\\
%		$g_R=90^\circ,g_L=0^\circ$& 4.041& 4.041$\pm$0.0054\\
		\hline
	\end{tabular}
	\caption{Cross sections and decay widths of different mixture angles of 0,30,45,60 and $90^\circ$ for $\wprime =220 GeV$ mass. \label{tab:mixingAngle} }
\end{table}

%There is a linear relation between the value of cross sections and the angle of $g_L$. With increasing the angle of $g_L$ from ziro to $ 90 ^\circ$, the cross section by 4.041(GeV) become 1.5 times larger than purely left-handed by 5.881(GeV). To illustrating the affect of changing mixture angle of $ g_R$ and  $g_L$ on amount of decay width and cross section for $ \wprime $ boson with the mass of 220 (GeV),

%It is noteworthy that all of our considerations about $ \wprime $ boson is based on the Sequential Standard Model (SSM)~\cite{Khachatryan:2014tva}. SSM is one of the extension of the Standard Model (SM) that contains extra heavy neutral bosons, i.e. $ Z' $ and $ \wprime $ with the same couplings to fermions and bosons as the standard Z and W bosons, of course with similar decay modes and branching fractions [ref]. The SSM $ \wprime $ boson does not interfere with the W boson. The coupling strength to fermions is as same as the W boson and decay widths of particles can determine by their mass. 


%We used TAUOLA package~\cite{Jadach:1990mz,Golonka:2003xt,Jadach:1993hs} for our simulation. TAUOLA is a library of Monte Carlo programs f/%or leptonic and semileptonic decays of $\tau$ lepton. Fully information about final state consist of neutrinos, distribution of moderator par%ticles and complete spin structure throughout the decay are some of the advantages of using this package.

%In experimental particle physic, experts have defined a concept called "Missing Transverse Energy", (MET), due to vendicate of conservational energy and momentum of a decay process. In hadron collider, the initial momentum of colliding partons along the beam axis is not known, therefore, calculation of total missing energy isnot possible, however, initial energy in particle travelling transverse energy to thebeam is zero. With this scenario, any net momentum in the transverse direction represents "MET". We calcute MET by looping over the both $\tau$ products.

As we discussed before, the final state of our signal include pure hadronic channel (\tauTau) and also a mixture of hadronic-leptonic channel (\lepTau ).  In figures \ref{fig:mt2} and \ref{fig:met}, the $MT_2$ and MET distributions for both channels in different \wprime masses are shown.
{\bf MET definition??} The transverse momentum of the leading and next-to-leading \Tau leptons in \tauTau channel are shown in figure \ref{fig:pt-hh}. The figure \ref{fig:pt-lh} shows the $P_T$ of the lepton and \Tau in \lepTau channel.
{\bf Figures are not consistent. (log, linear,...) The colors and legends are not visible. axis title should be fixed in pt distributions.}
\begin{figure}[!ht]
\centering
%\subfigure[{MET-hadronic channel}]{\label{met-hh}
\includegraphics*[width=.45\textwidth]{figs/MET-hh.pdf}
\hspace{3mm}
%\subfigure[{MET-$\tau$ lepton channel}]{\label{met-lh}
\includegraphics*[width=.45\textwidth]{figs/MET-lh.pdf}
\caption{Missing transverse energy in different masses of \wprime }
\label{fig:met}
\end{figure}
\begin{figure}[htb]
\centering
%\subfigure[{$MT_2$ - hadronic channel}]{\label{mt2-hh}
\includegraphics*[width=.45\textwidth]{figs/MT2-hh.pdf}
\hspace{3mm}
%\subfigure[{$MT_2$ - $\tau$ lepton channel}]{\label{mt2-lh}
\includegraphics*[width=.45\textwidth]{figs/MT2-lh.pdf}
\caption{$MT_2$ in different masses of \wprime }
\label{fig:mt2}
\end{figure} 
\begin{figure}[!ht]
\centering
%\subfigure[{$P_T {\tau}$ maximum-hadronic channel}]{\label{pt-max}
\includegraphics*[width=.45\textwidth]{figs/PT-max.pdf}
\hspace{3mm}
%\subfigure[{$P_T {\tau}$ minimum - $\tau$ lepton channel}]{\label{pt-min}
\includegraphics*[width=.45\textwidth]{figs/PT-min.pdf}
\caption{$P_T {\tau}$ maximum and minimum in different masses of \wprime }
\label{fig:pt-hh}
\end{figure}
\begin{figure}[!ht]
\centering
%\subfigure[{$P_T {\tau}$ -$\tau$ lepton channel}]
\includegraphics*[width=.45\textwidth]{figs/PT-lh.pdf}
\hspace{3mm}
%\subfigure[{$P_T {leptons}$ - $\tau$ lepton channel}]{\label{pt-ll}
\includegraphics*[width=.45\textwidth]{figs/PT-ll.pdf}
\caption{$P_T^{\Tau}$ and $P_T^{\ell}$ in different masses of \wprime  }
\label{fig:pt-lh}
\end{figure}

  
