\section{SIMULATION}\label{sec:simulation}

%{\bf The text should be updated to address the following items discussed in the meeting.}
%4 [Zamiri] Simulation of signal events 
%4.1 madgraph {\bf Done}
%4.2 reference to the wprime package that used {\bf Done}
%4.3 quote coupling value {\bf Done}
%4.3.1 doubled and halved {\bf Done}
%4.4 explain about the mixture of left-and right handed {\bf Done}
%4.5 decay width and validation (ref to the W->tb) {\bf Done}
%4.6 cross sections (mass, mixture angle) {\bf Done}
%4.7 discussion about the branching fractions {\bf Done??}
%4.7.1 assume SM values {\bf Done}
%4.7.2 assume that it prefers tau between leptons
%4.8 validation plots ??? / kinematic distributions 
%4.8.1 gen mt2 for different masses {\bf Done}
%4.8.2 gen tau-pt for different masses {\bf Done}
%4.9 yields and final efficiencies 
%4.9.1 2 tables for SM coupling and double coupling (for different masses)
%{\bf Done} 
%4.9.2 1 table for SM coupling and different mixing angles = (0,30,45,60) (for mass=220GeV) {\bf Done}

To generate the signal events  MadGraph5\_aMC@NLO~\cite{Alwall:2014hca} package is used which is the new version of MadGraph5~\cite{Alwall:2011uj} matrix-element generator. In continue, we refer to this event generator  as MadGraph.
The model used for signal generation is \wprime effective model (WEff-UFO)~\cite{Sullivan:2002jt}, which  is an extension of SM by adding \wprime interactions with the SM fermions.
The Universal FEYNRULES\cite{Christensen:2008py} Output format (UFO) is a flexible description of new models providing the abstract information about the particles, parameters and vertices of a new model that can be used by matrix-element generators. 
The signal of our interest is $ pp\rightarrow W^{\prime+} W^{\prime-}$. 
For each set of parameters, 10000 events are generated at a center-of-mass energy of 8 TeV. We considered purely left-handed  \wprime ($ g_L = g_{SM}$ and $ g_R $ = 0). 
This means interactions with quarks and leptons are both allowed. Different masses of  \wprime boson in the range of 100 through 400 GeV are used in this analysis.  
Decay width or life time of \wprime boson depends on decay modes, coupling strength of the decay process, and kinematic constraints. Decay widths corresponding to each mass of \wprime boson are estimated  by MadGraph. The results agree with the values reported in Ref.\cite{Sullivan:2002jt} where the total width of \wprime and partial width of $\wprime \rightarrow t \bar{b},\bar{t}b $ are calculated in the leading order and the next to leading order precision.
The production cross section and total width for the decay of \wprime  to both quarks and leptons in different masses are listed in table \ref{tab:Xsec,L-h}. The values are for  proton-proton (pp) collisions at $\sqrt{s}$ = 8 TeV.
 \begin{table}[!htb]
	\centering
\begin{tabular}{|c|c|c|}
\hline 
\wprime mass (GeV)  &  Decay width (GeV)  &  Cross section (fb)\\
\hline 
100 & 2.509 & 970  \\
130 & 3.262 & 344 \\
160 & 4.014 &147 \\
190 & 4.838 &69.3 \\
 220& 5.881 & 33.4 \\
 250 &6.989 &17.2 \\
 280 &8.104 &9.62 \\
 310 &9.211 &5.64 \\
 340 &10.31 &3.46 \\
 370 &11.39 & 2.21\\ 
 400 &12.46 &1.45\\
\hline
\end{tabular}
\caption{Cross sections and decay widths when $ g_R=0 , g_L= g_{SM} = 0.65 $  for various \wprime masses for pp collisions at $\sqrt{s}$ = 8 TeV. The relative uncertainties on the cross sections are less than 1\%. \label{tab:Xsec,L-h} }
\end{table}

As a cross check the branching fraction which is defined as the partial decay width to a special channel divided by the total width is compared for both signs of \wprime. 
It can be observed in tables \ref{tab:W'Plus} and \ref{tab:W'Minus} that the values are consistent for \wprimep and \wprimem. 
\begin{table}[htb]
	\centering
	\begin{tabular}{|c|c|c|}
		\hline 
		\wprime mass (GeV)  &  BR \wprimep $\rightarrow \tau,\nu_\tau $& BR  \wprimep$\rightarrow  t \bar{b}$ \\
		\hline 
		200 & 0.08 & 0.03\\
		400 &0.08& 0.19\\
		600 &0.08&0.22\\
		800&0.08 &0.23\\
		\hline
	\end{tabular}
	\caption{Branching fractions of \wprimep ~when $ g_R=0 , g_L= g_{SM} = 0.65 $ for various \wprimep ~masses. \label{tab:W'Plus} }
\end{table}
\begin{table}[htb]
	\centering
	\begin{tabular}{|c|c|c|}
		\hline 
		\wprime Mass (GeV)  &  BR  \wprimem$\rightarrow \tau,\nu_\tau $ & BR  \wprimem$\rightarrow  \bar{t}b $ \\
		\hline 
		200 &  0.09 &0.03 \\
		400  &0.08 &0.19 \\
		600  &0.08&0.22\\
		800  &0.08 &0.23 \\
		\hline
	\end{tabular}
		\caption{Branching fractions of \wprimem  ~when $g_R=0 , g_L= g_{SM} = 0.65 $ for various \wprimem ~masses. \label{tab:W'Minus} }
\end{table}
%In 8-9\% of the cases \wprime boson decays to $\tau$ lepton and the corresponding neutrino and in about 27\% of cases it decays to top and  bottom quark. Due to the fact that by choosing mass range of 200 to 400 (GeV), decay of $ \wprime^\pm $ boson to top quarks is kinematically allowed, the decay of the $ \wprime\rightarrow \tau,\nu_\tau $ is almost constant with increasing the \wprime mass.

The couplings of \wprime are not fixed by the model, so to investigate the effect of the couplings, we calculate the production cross section and decay width when the couplings are multiplied by 2 or 0.5. As can be seen in tables \ref{tab:Xsec,half} and \ref{tab:Xsec,twice}, the cross section is scaled by 16 and 1/16 when the coupling is doubled and halved, respectively. It is noticeable that the values of decay widths are proportional to  factors of 4 and  1/4 for the doubled and halved left-handed coupling values. The behavior of the cross sections and decay widths is consistent with our expectation.
\begin{table}[htb]
	\centering
\begin{tabular}{|c|c|c|}
\hline 
\wprime Mass (GeV)  &  Decay Width (GeV) &  Cross section(fb)\\
\hline 
  100& 0.627& 60.8\\
  130& 0.815& 21.6\\
  160& 1.004& 9.25\\
  190& 1.209& 4.35\\
  220& 1.470& 2.08\\
  250& 1.747 &1.08\\
  280& 2.026& 0.599\\
  310& 2.303& 0.351\\
  340& 2.577& 0.217\\
  370& 2.848& 0.138\\
  400& 3.116 & 0.0912\\ 
\hline
\end{tabular}
\caption{Cross sections and decay widths when $ g_R=0 , g_L=g_{SM}/2 =0.32$  for various \wprime masses. \label{tab:Xsec,half} }
\end{table}  
\begin{table}[htb]
	\centering
\begin{tabular}{|c|c|c|}
\hline 
\wprime Mass (GeV)  &  Decay Width (GeV) &  Cross section(fb)\\
\hline 
100& 10.04& 15100\\
130& 13.05& 5300\\
160& 16.06& 2260 \\
190& 19.35& 1060\\
220& 23.53& 508\\
250& 27.95& 262\\
280& 32.41& 145 \\
310& 36.84& 85.2 \\
340& 41.22& 52.2\\
370& 45.56& 33.2 \\
400& 49.86& 21.9 \\ 
\hline
\end{tabular}
\caption{Cross sections and decay widths in the case that $ g_R=0 , g_L=2\times g_{SM} =1.30$  for various \wprime masses. \label{tab:Xsec,twice} }
\end{table}

In an alternative approach, the left-handed and right-handed couplings are changed in a way that their squared sum is constant and equal to $g_{SM}^2$.
\begin{equation}
g_{SM}^2 = (g'_L)^2 +  (g'_R)^2 
\end{equation}
It is easier, to define a mixing angle and rewrite the couplings as:
\begin{eqnarray}
g'_L  = g_{SM} cos\theta \\
g'_R  = g_{SM} sin\theta
\end{eqnarray}
By varying $\theta$ from 0 to $ 90^\circ $, the \wprime goes from a purely left-handed to purely right-handed vector boson. The latter \wprime does not have any interaction with the leptons. 
The variation of cross sections and decay widths due to various mixing angles, for \wprime boson with a mass of 220 GeV is shown in table \ref{tab:mixingAngle}.
\begin{table}[htb]
	\centering
	\begin{tabular}{|c|c|c|c|}
		\hline 
		Mixing angle $\theta$  &  Decay width (GeV) & BR(\wprime $\rightarrow \tau \nu_\tau$)  &  Cross section (fb)\\
		\hline 
		0      &          &          & \\
		30      &          &          & \\
		45      &          &          & \\
		60      &          &          & \\
		90      &          &          & \\
%		$g_R=45^\circ,g_L=45^\circ$& 4.997& 4.99$\pm$0.0053\\
%		$g_R=30^\circ,g_L=60^\circ$ &5.452& 5.452$\pm$0.0054\\
%		$g_R=60^\circ,g_L=30^\circ$& 4.532& 4.532$\pm$0.0053 \\
%		$g_R=0^\circ,g_L=90^\circ$&5.881& 5.881$\pm$0.0059\\
%		$g_R=90^\circ,g_L=0^\circ$& 4.041& 4.041$\pm$0.0054\\
		\hline
	\end{tabular}
	\caption{Decay widths, branching ratios and cross sections  of \wprime boson with 220 GeV mass for different mixing angles. \label{tab:mixingAngle} }
\end{table}

As we discussed before, the final state of our signal include pure hadronic channel (\tauTau) and also a mixture of hadronic-leptonic channel (\lepTau ).  In figures \ref{fig:met} and \ref{fig:mt2}, the $MT_2$ and \MET  ~distributions for both channels in different \wprime masses are shown.
\begin{figure}[!htb]
\centering
\includegraphics*[width=.45\textwidth]{figs/MET-hh.pdf}
\hspace{3mm}
\includegraphics*[width=.45\textwidth]{figs/MET-lh.pdf}
\caption{Missing transverse momentum (\MET) for different masses of \wprime. The events of \tauTau(\lepTau) channel are shown in left (right).}
\label{fig:met}
\end{figure}
\begin{figure}[htb]
\centering
\includegraphics*[width=.45\textwidth]{figs/MT2-hh.pdf}
\hspace{3mm}
\includegraphics*[width=.45\textwidth]{figs/MT2-lh.pdf}
\caption{$M_{T2}$ for different masses of \wprime. The events of \tauTau (\lepTau) channel are shown in left (right).}
\label{fig:mt2}
\end{figure} 
The transverse momentum of the leading and next-to-leading \Tau leptons in \tauTau channel are shown in figure \ref{fig:pt-hh}. The figure \ref{fig:pt-lh} shows the \pt ~of the lepton and \Tau in \lepTau channel.
{\bf Figures are not consistent. (log, linear,...) The colors and legends are not visible. axis title should be fixed in pt distributions.}
\begin{figure}[!ht]
\centering
\includegraphics*[width=.45\textwidth]{figs/PT-max.pdf}
\hspace{3mm}
\includegraphics*[width=.45\textwidth]{figs/PT-min.pdf}
\caption{The maximum  and minimum of $\pt^{\Tau}$ in \tauTau ~channel for different masses of \wprime.}
\label{fig:pt-hh}
\end{figure}
\begin{figure}[!ht]
\centering
\includegraphics*[width=.45\textwidth]{figs/PT-lh.pdf}
\hspace{3mm}
\includegraphics*[width=.45\textwidth]{figs/PT-ll.pdf}
\caption{Left (right) plot shows $\pt^{\Tau}$ ($\pt^{\ell}$) in \lepTau ~channel for different masses of \wprime.}
\label{fig:pt-lh}
\end{figure}

  
