\section{SIMULATION}\label{sec:evo}

{\bf The text should be updated to address the following items discussed in the meeting.}

4 [Zamiri] Simulation of signal events

4.1 madgraph

4.2 reference to the wprime package that used

4.3 quote coupling value

4.3.1 doubled and halfed

4.4 explain about the mixture of left-and right handed

4.5 decay width and validation (ref to the W->tb)

4.6 cross sections (mass, mixture angle)

4.7 discussion about the branching fractions

4.7.1 assume SM values

4.7.2 assume that it prefers tau between leptons

4.8 validation plots ??? / kinematic distributions

4.8.1 gen mt2 for different masses

4.8.2 gen tau-pt for different masses

4.9 yields and final efficiencies

4.9.1 2 tables for SM coupling and double coupling (for different masses)

4.9.2 1 table for SM coupling and different mixing angles = (0,30,45,60) (for mass=220GeV)

\section{SIMULATION}\label{sec:evo}
  {\small MADGRAPH 5} is the new version of {\small MADGRAPH} matrix generator [ref]. We used MadGraph [ref] for simulating signal event beacause implementation of accurate simulation for signal is based on the ability of {\small MADGRAPH}. It is the tool with stronge power of extracting physics from the data and making precise prediction for physics Beyond the Standard Model {\small (BSM)}. Generating tree-level matrix elements and measurement of cross sections for classes of renormalizable models and establishment of {\small FEYNRULES} [ref] to perform new physics models and finaly making links between {\small FEYNRULES} and matrix elements and standardizing them by Universal F{\small EYNRULES} Output format, the {\small (UFO)} [ref], are some of the significant benefits of {\small MADGRAPH 5}. {\small UFO} is the flexible format of models goes beyond existing formats in order to provide the abstract information about the particles, parameters and vertices of the model that can be translate into {\small PYTHON} [ref] and achieve by matrix-element generators. 
 
 The model performed for signal generation was $\wprime$ Effective model {\small (WEff-UFO)}, that is an extension of Standard Model {\small (SM)} and it is available in [ref]. One can also achieve to example of Mathematica notebook that loads the model, calculates the Feynman rules and extract the model files within the UFO format in [ref]. The \wprime Effective model authenticated the using MadGraph searches consist of $\wprime -> tb$ vertex at the LHC [ref].
 
  As it is mentioned in last section, the signal of interest is $ pp\rightarrow {\tiny \wprime \wprime} $. We calculate cross section of signal by considering different parameters in generation process and determine measure of increasing or suppressing of cross section after changing parameters. The generation implement with 10000 events and at a center-of-mass energy of $ \sqrt{s}=8 $ TeV in each run. We considered purely left-handed coupling  for $ g_L $ and $ g_R $ values.  The right-handed (left-handed) \wprime-boson couplings to quarks and leptons are considered to be $g'_R = 0$ and $g'_L = g_{(SM)}$ respectively, in other words decays to quarks and leptons are both included. A variety masses of \wprime boson in range of 100 through 400 GeV were used for measuring cross sections.  Decay width or life time is depends on decay modes or channels has a crucial role in conservation laws for appropriate quantum numbers, coupling strength of the decay process, and kinematic constraints. Decay widths corresponding to each mass of $ \wprime $ estimated concurrently by {\small MADGRAPH 5} and those calculated with this model compered to [ref] which investigated the LO and NLO partial and total width according to their branching fractions and cross sections for  $ \wprime_{R,L}\rightarrow t \bar{b},\bar{t}b $.  There is a good agreement among different values.  In table I we list the cross sections and total width for the decay of $ \wprime $ to both quarks and leptons in different mass. 

 \begin{table}[htb]
	\centering
\begin{tabular}{|c|c|c|}
\hline 
\wprime Mass(GeV)  &  Decay Width  &  Cross section(pb)\\
\hline 
100 & 2.50885 & 0.9696$\pm$0.003  \\
130 & 3.26166& 0.3441$\pm$0.0012 \\
160 & 4.014482 &0.1467$\pm$0.0006 \\
190 & 4.837699 &0.06932$\pm$0.00027 \\
 220& 5.881464& 0.03341$\pm$0.00014 \\
 250 &6.988690 &0.0172$\pm$6.01e-05 \\
 280 &8.103721 &0.00962$\pm$3.8e-02 \\
 310 &9.210820 &0.00564$\pm$2.4e-05 \\
 340 &10.30623 &0.003462$\pm$1.3e-05 \\
 370 &11.39039& 0.002213$\pm$8.9e-06\\ 
 400 &12.46427 &0.001451$\pm$5.5e-06\\
\hline
\end{tabular}
\caption{Cross sections and decay widths in the case that $ g_R=0 , g_L=0.458486075  $ corresponding to various \wprime masses. \label{tab1} }
\end{table}

 

 
 In order to do a precise evaluation of cross sections sensibility under changes of coupling values, we estimated once half and again doubled left-handed coupling values. As can be seen in table 2 and 3, the amounts of cross section becamed 16 times ($ 1/16 $) larger (smaller) than cross sections in table I respectively. It is noticeable that the values of decay widths are proportional to the factor of 4 and ($ 1/4 $) for doubled (half) left-handed coupling values.
  \begin{table}[htb]
	\centering
\begin{tabular}{|c|c|c|}
\hline 
\wprime Mass (GeV)  &  Decay Width  &  Cross section(pb)\\
\hline 

 100& 0.62721& 0.06084$\pm$0.00018\\
  130& 0.81542& 0.02156$\pm$7e-05\\
  160& 1.00362& 0.009247$\pm$3.7e-05\\
  190& 1.20944& 0.004351$\pm$1.8e-05\\
  220& 1.47035& 0.00208$\pm$8.3e-06\\
  250& 1.74718 &0.001077$\pm$4.4e-06\\
  280& 2.02593& 0.0005993$\pm$2.2e-06\\
  310& 2.30270& 0.000351$\pm$1.3e-06\\
 340& 2.57655& 0.0002166$\pm$8.3e-07\\
 370& 2.84757& 0.0001383$\pm$5e-07\\
  400& 3.11605& 9.119e-05$\pm$3.1e-07\\ 

\hline
\end{tabular}
\caption{Cross sections and decay widths in the case that $ g_R=0 , g_L=0.3241986  $ (half of the standard model) for various \wprime masses. \label{tab1} }
\end{table}
  
 \begin{table}[htb]
	\centering
\begin{tabular}{|c|c|c|}
\hline 
\wprime Mass (GeV)  &  Decay Width  &  Cross section (pb)\\
\hline 
100& 1.003553e+01& 15.08$\pm$0.045\\
130 &1.304687e+01& 5.297$\pm$0.017\\
160& 1.605799e+01& 2.26$\pm$0.0072 \\
190&1.93512e+01& 1.064$\pm$.0025\\
220& 2.352564e+01& 0.5078$\pm$0.0014\\
250& 2.795486e+01& 0.2619 $\pm$ 0.00072\\
280& 3.241489e+01& 0.1451 $\pm$ 0.00043\\
310& 3.684301e+01& 0.08521 $\pm$ 0.0002\\
340& 4.122483e+01& 0.05223$\pm$0.00013\\
370& 4.556095e+01&0.03324 $\pm$ 9.314e-05\\
400& 4.985667e+01& 0.02192 $\pm$ 5.1e-05\\ 

\hline
\end{tabular}
\caption{Cross sections and decay widths in the case that $ g_R=0 , g_L=1.2967944  $ (twice of the standard model) for various \wprime masses. \label{tab1} }
\end{table}
In physics, a coupling constant or gauge coupling parameter is a number that determines the strength of the force exerted in an interaction[ref]. In this paper, we deemed a new concept "mixture angle" for coupling constants in order to study variation of cross sections according to various mixture angles, that means, for purely right-handed and left-handed couplings, the angle between $ g_R$ and  $g_L$ considered to be $ 90 $ and $ 0 ^\circ$ and vice versa respevtively, the other angles between these two variables supposed to be $ 0, 30, 45, 60 $ and $ 90^\circ $. In other words, we can imagine a two-dimentional coordinate system that vector X and Y represent the angle of $ g_R$ and  $g_L$ coupling constants. There is a linear relation between the value of cross sections and the angle of $g_L$. With increasing the angle of $g_L$ from ziro to $ 90 ^\circ$, the cross section by 4.041(GeV) become 1.5 times larger than purely left-handed by 5.881(GeV). To illustrating the affect of changing mixture angle of $ g_R$ and  $g_L$ on amount of decay width and cross section for $ \wprime $ boson with the mass of 220 (GeV), see table IV which displays the results of these changes.

It is noteworthy that all of our considerations about $ \wprime $ boson is based on the Sequential Standard Model (SSM). SSM is one of the extension of the Standard Model (SM) that contains extra heavy neutral bosons, i.e. $ Z' $ and $ \wprime $ with the same couplings to fermions and bosons as the standard Z and W bosons, ofcourse with similar decay modes and branching fractions [ref]. The SSM $ \wprime $ boson does not interfere with the W boson. It has the same coupling strength to fermions as the W boson and its decay width is determined by its mass. 

  The branching fraction is the fraction of events for a particular particle measured to decay in a certain way. The branching fraction is defined as the partial decay width divided by the total width. It is important to meseare the branching fractions of $ pp\rightarrow \wprime$ between these two channels : $ \wprime\rightarrow \tau,\nu_\tau $ and $ \wprime\rightarrow  t \bar{b},\bar{t}b $, due to comparing the percentage of decays of $ \wprime $ boson to $ \tau $ and top quark in different channels for $ \wprime^\pm $ boson. We performed this comparison for masses 200, 400, 600 and 800 GeV. As it is observable in table \ref{tab:W'Plus} and \ref{tab:W'Minus}, in 8 or $9\%$ of cases \wprime boson decaye to tau and a neutrino tau and in about $27\%$ of cases it decays to top and a bottom quark. Due to the fact that by choosing mass range of 200 to 400 (GeV), decay of $ \wprime^\pm $ boson to top quarks is allowed, the decay of the $ \wprime\rightarrow \tau,\nu_\tau $ is almost consistant even with growing the mass. All of these determinations occured for the situation that the mixture angle of $ g_R$ and  $g_L$ is assumed to be $45^\circ $. 
 



\begin{table}[htb]
	\centering
\begin{tabular}{|c|c|c|}
\hline 
Mixture angle of $ g_R \& g_L $  &  Decay Width  &  Cross section (pb)\\
\hline 
$g_R=45^\circ,g_L=45^\circ$& 4.997125& 4.99$\pm$0.0053\\
$g_R=30^\circ,g_L=60^\circ$ &5.452345& 5.452$\pm$0.0054\\
$g_R=60^\circ,g_L=30^\circ$& 4.532335& 4.532$\pm$0.0053 \\
$g_R=0^\circ,g_L=90^\circ$&5.881464& 5.881$\pm$0.0059\\
$g_R=90^\circ,g_L=0^\circ$& 4.041445& 4.041$\pm$0.0054\\


\hline
\end{tabular}
\caption{Cross sections and decay widths of different mixture angles of 0,30,45,60 and $90^\circ$ for $\wprime =220 GeV$ mass. \label{tab:220 GeV} }
\end{table}


 
 
 \begin{table}[htb]
	\centering
\begin{tabular}{|c|c|c|}
\hline 
\wprime Mass (GeV)  &  BR $ W'^+\rightarrow \tau,\nu_\tau $& BR $ W'^+\rightarrow  t \bar{b},\bar{t}b $ \\
\hline 
200 & 0.082610\% & 0.03379\%\\
400 &0.07665\%& 0.19082\%\\
600 &0.07701\%&0.22118\%\\
800&0.07799\% &0.23229\%\\

\hline
\end{tabular}
\caption{Cross sections and branching fractions of $W'^+$ in the case that $ g_R=0.458486057 , g_L=0.458486057  $ for various $W'^+$ masses. \label{tab:W'Plus} }
\end{table}


 \begin{table}[htb]
	\centering
\begin{tabular}{|c|c|c|}
\hline 
\wprime Mass (GeV)  &  BR $ W'^-\rightarrow \tau,\nu_\tau $ & BR $ W'^-\rightarrow  t \bar{b},\bar{t}b $ \\
\hline 
200 &  0.08682\% &0.03368\% \\
400  &0.07847\% &0.191023\% \\
600  &0.07827\%&0.22194\%\\
800  &0.079037\% &0.23348\% \\

\hline
\end{tabular}
\caption{Cross sections and branching fractions of $W'^-$ in the case that $ g_R=0.458486057 , g_L=0.458486057  $ for various $W'^-$ masses. \label{tab:W'Minus} }
\end{table}






